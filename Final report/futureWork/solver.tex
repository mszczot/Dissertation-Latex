\subsection{Functionality}
All the requirements identified in the design stage that were not implemented in this project should be considered as future work. Those include computing additional semantics, such as ground, ideal, semi-stable. Furthermore, ALIAS could be extended to give user possibility of removing arguments and attacks. 

\subsection{Solver}
ALIAS is using PicoSAT solver with PycoSAT python module as the translation between Python and PicoSAT. As discussed in section \ref{section:pycosat}, PicoSAT is a high performance solver, which helps to increase the performance of ALIAS itself. However, it has certain limitations. The most noticeable one is the limited ability of encoding the clauses. PycoSAT only allows the clauses to be encoded in the format of conjunctions of disjunctions. This can have big impact on the way semantics are being encoded. Future work might include investigation of alternative solvers with Python interfaces and incorporating them in ALIAS. One of the most noticeable alternatives is a Satisfiability Modulo Theories solver from Microsoft Research, Z3 Prover. Z3 provides an API in Python by default, hence no additional packages will be required. Z3 supports standard Boolean operators: \textit{and}, \textit{or}, \textit{implies}, \textit{if-then-else} and does not limit clauses to disjunctions, but allows the individual clauses to be a selection of conjunctions. 

\begin{figure}
	\begin{minted}{python}
	>>> import z3
	>>>
	>>> a = Bool('a')
	>>> b = Bool('b')
	>>> c = Bool('c')
	>>> solve(Implies(a, c), Or(not(b), c), And(a, c))
	\end{minted}
	\caption{Example of Z3Py syntax}
	\label{fig:z3}
\end{figure}

Furthermore, Z3Py can introduce clearer syntax when compared to PycoSAT, as can be seen in figure \ref{fig:z3}. Improved syntax of the clauses would improve the readability and maintenance of ALIAS. 

\subsubsection{Heuristics}
Further performance improvements can be done using certain heuristics when encoding the individual semantics. By using specific properties of the arguments, argumentation frameworks and semantics there are number of optimization that can be applied to reduce the solvers' search space and improve performance.

If we consider argumentation framework from figure \ref{fig:argumentationFrameworkFigure}, we can see that complete, preferred and stable extensions are \textit{(a,d)} and are all the same. Argument \textit{b} cannot be part of any solution as is not admissible, i.e. there are no arguments in the argumentation framework that attack argument \textit{a}, the only attacker of \textit{b}. Similarly, we can see that argument \textit{c} cannot be a part of any solution as is not admissible either. Hence, those arguments and any solution including them can be safely removed from the SAT solver models. 

With small argumentation framework the impact on performance might not be noticeable. However, when the search space is large as it is with bigger argumentation framework, reducing it by any means can help to achieve faster solving times. This is also a single example how heuristics can be used. Future work can involve identifying and implementing properties to exclude or include certain arguments.

\subsubsection{Graph pattern recognition}
Another interesting path for future work is to investigate the possibility to use the pattern recognition within the argumentation framework graph. Graph matching techniques have been around for over 30 years and have been successfully used within the image and video analysis, document processing, biometric identification, image databases and biological and biomedical applications \citep{graphMatching}.

\begin{figure}[h]
	\tikzset{
		main/.style={draw, rectangle, rounded corners, minimum height=1cm, minimum width=4.5cm},
		arrow/.style={thick,<-,>=stealth}
	}
	\centering
	\begin{tikzpicture}[auto,node distance=1.5cm]

	\coordinate(x);
	\node[draw=none,fill=none][above=0.75cm of x](d){d};
	\node[draw=none,fill=none][below=0.75cm of x](e){e};
	\node[draw=red,fill=none][right=of x](a){a};
	\node[draw=red,fill=none][right=of a](b){b};
	\node[draw=red,fill=none][right=of b](c){c};	
	\coordinate[right=of c](z);
	\node[draw=none,fill=none][above=0.75cm of z](f){f};
	\node[draw=none,fill=none][below=0.75cm of z](g){g};
	%%% ARROWS %%%
	\draw[arrow](a) -- (d);
	\draw[arrow](a) -- (e);
	\draw[arrow](b) -- (a);
	\draw[arrow](c) -- (b);
	\draw[arrow](f) -- (c);
	\draw[arrow](g) -- (c);
	
	\draw[thick,<-,>=stealth,transform canvas={xshift=-0.2em}](f) -- (g);
	\draw[thick,<-,>=stealth,transform canvas={xshift=0.2em}](g) -- (f);
	
	\draw[thick,<-,>=stealth,transform canvas={xshift=-0.2em}](d) -- (e);
	\draw[thick,<-,>=stealth,transform canvas={xshift=0.2em}](e) -- (d);
	\end{tikzpicture}
	\caption{Argumentation Framework example for Graph Pattern Matching}
	\label{fig:graphPatternMatching}
\end{figure}

In terms of argumentation frameworks, graph pattern matching could be used to limit the number of variables used by the solver. Lets consider the argumentation framework from figure \ref{fig:graphPatternMatching}. Highlighted arguments \textit{a}, \textit{b} and \textit{c} create a quite common pattern in any abstract argumentations: \textit{a} attacks \textit{b} and \textit{b} attacks \textit{c}. Hence, if we replace those three arguments with a single node, as shown in figure \ref{fig:graphPatternMatchingExtracted}, we can reduce the number of variables for the solver, thus reducing the search space. During the search for the solution \textit{abc} can be treated as a single argument and expanded once the solutions are provided. The outcome of the sub-arguments of \textit{abc} will depends whether \textit{abc} is defined as \textit{in} or \textit{out}:

\begin{enumerate}
	\item \textit{abc} is \textit{in} - if \textit{abc} is a part of the solution, then it will be replaced by arguments \textit{a} and \textit{c}
	\item \textit{abc} is \textit{out} - if \textit{abc} is not part of the solution, then it will be replaced by a single argument \textit{b}
\end{enumerate}  

\begin{figure}[h]
	\tikzset{
		main/.style={draw, rectangle, rounded corners, minimum height=1cm, minimum width=4.5cm},
		arrow/.style={thick,<-,>=stealth}
	}
	\centering
	\begin{tikzpicture}[auto,node distance=1.5cm]
	
		\node[draw=red,fill=none][right=of x](a){a};
		\node[draw=red,fill=none][right=of a](b){b};
		\node[draw=red,fill=none][right=of b](c){c};
		\draw[arrow](b) -- (a);
		\draw[arrow](c) -- (b);
		
		\coordinate[below=of b](space);
		\coordinate[below=0.75cm of space](space1);
		\coordinate[left=2cm of space1](x);
		\node[draw=none,fill=none][above=0.75cm of x](d){d};
		\node[draw=none,fill=none][below=0.75cm of x](e){e};	
		\node[draw=none,fill=none][right=of x](abc){abc};
		\coordinate[right=of abc](z);
		\node[draw=none,fill=none][above=0.75cm of z](f){f};
		\node[draw=none,fill=none][below=0.75cm of z](g){g};
		%%% ARROWS %%%
		\draw[arrow](abc) -- (d);
		\draw[arrow](abc) -- (e);
		\draw[arrow](f) -- (abc);
		\draw[arrow](g) -- (abc);
		
		\draw[thick,<-,>=stealth,transform canvas={xshift=-0.2em}](f) -- (g);
		\draw[thick,<-,>=stealth,transform canvas={xshift=0.2em}](g) -- (f);
		
		\draw[thick,<-,>=stealth,transform canvas={xshift=-0.2em}](d) -- (e);
		\draw[thick,<-,>=stealth,transform canvas={xshift=0.2em}](e) -- (d);

	\end{tikzpicture}
	\caption{Graph Matching Pattern extracted}
	\label{fig:graphPatternMatchingExtracted}
\end{figure}

This approach can be useful, especially with semantics that tries to maximize the solution, like stable and preferred extensions. By identifying certain patterns, and replacing them in the argumentation frameworks, the search space can be significantly reduced. This will allow for smaller sets of possible solutions to be created and improved performance. 