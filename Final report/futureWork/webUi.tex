\section{Web User Interface}

\subsection{Usability}
In terms of usability, ALIAS Web Interface can be extended to bind mouse events on the argumentation framework graph and bind them to certain functionalities of ALIAS. For example, double tap on the white space of the page could add a new argument, while clicking and holding the mouse button on argument could start drawing an attack. 

Changing the interface to a standard editor interface would also help a user to learn it faster and be able to perform the operations with ease. Introducing toolbar with standard icons, for example, similar to the Gliffy interface shown in figure \ref{fig:toolbarExample}, could further enhance the usability.

\begin{figure}
	\includegraphics[width=\textwidth]{toolbar_example}
	\caption{Gliffy toolbar}
	\label{fig:toolbarExample}
\end{figure}

\subsection{Functionality}
Web Interface should fully extend the capabilities of ALIAS. Functionality to check if the given argument is credulously or skeptically accepted should be available through the interface. Furthermore, if ALIAS capabilities will be extended in the future, this should also be reflected in Web UI.

Another feature that could be added is to allow a user to view properties of each node and edge in the side panel. This would give an insight of elements in displayed argumentation framework. Furthermore, a section of the argumentation framework could be selected and details viewed in the properties panel. This could be used for 'locking' certain arguments, where ALIAS would return only semantics with selected arguments.

