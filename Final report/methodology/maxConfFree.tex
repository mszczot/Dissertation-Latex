\subsection{Maximal Conflict Free Sets}
According to the definitions \ref{def:preferredExtension} and \ref{def:stableExtension}, Preferred and Stable extensions are based on the maximal sets. Preferred extensions are the maximal element with respect to the set from all the admissible sets, and Stable extensions are the maximal conflict free sets that attack every argument outside those sets. Furthermore, since each admissible set is also a conflict free set, this property has been used in exploring possible solutions for computing Preferred and Stable extensions.

Let \textit{args} be a collection of arguments in a given argumentation framework \textit{AF}. Then \textit{args} can be represented as $args=arg_1, arg_2,...,arg_n$ where \textit{n} is the total number of arguments in \textit{AF}. 
Furthermore, set of arguments \textit{CF} is conflict free, if and only if there are no attacks between the arguments within the set \textit{CF}: $\forall a,b \in AF, (a, b) \notin attacks$.

Hence, the maximal conflict free sets can be represented as:
\begin{equation}
\begin{array}{lr}
args_1 $ = $ \text{args} $ - $ \text{a} \\
args_2 $ = $ \text{args} $ - $ \text{b} \\
\end{array} \bigg| \forall a, b \in AF, (a,b) \in attacks
\end{equation}

Based on the proposition above, the maximal conflict free set can be calculated by splitting the existing sets of arguments within the given argumentation framework by each pair of arguments from the \textit{attacks}. The process is shown in example in figure \ref{fig:mcfExample}. The Argumentation Framework \textit{AF} consist of four arguments: a, b, c, d, and 3 attacks: a attacks b, b attacks c, and d attacks c. This argumentation framework is presented in figure \ref{fig:af1}. 

\begin{figure}[h]
	\centering
	\includegraphics[width=\linewidth]{"img/mcf "}
	\caption{Maximal Conflict Free sets creation example}
	\label{fig:mcf}
\end{figure}

Firstly, the attack \textit{(a,b)} is applied to the set of all arguments - \textit{\{a,b,c,d\}}, causing the set of arguments to split into two separate sets: one that does not contain argument \textit{b} - \textit{\{a,c,d\}}; and another that does not contain argument \textit{a} - \textit{\{b,c,d\}}. Then each attack from the argumentation framework is applied to the newly created sets, splitting them further into conflict free sets. Attack \textit{(b,c)} has impact only on the second set - \textit{\{b,c,d\}}, since argument \textit{b} is removed from the first set. Hence, applying the second attack to the sets, creates 3 sets: \textit{\{a,c,d\}}, \textit{\{c,d\}} and \textit{\{b,d\}}. Last attack - \textit{(c,d)}, has only impact on the first and second sets from the last step, giving the final set of Maximal Conflict Free sets of \textit{\{a,c\}}, \textit{\{a,d\}} and \textit{\{b,d\}}.

\begin{algorithm}
	\caption{Maximal Conflict Free sets calculation}\label{mcfPseudocode}
	\nl \textit{args} $\gets$ \textit{arguments from AF}\;
	\nl \textit{attacks} $\gets$ \textit{attacks of AF}\;
	\nl \textit{conflictFreeSets} \;
	\nl\ForEach{\textit{attack} $\in$ \textit{attacks}}
	{
		\If{\textit{conflictFreeSets} is not empty}{
				\nl \ForEach{set $\in$ conflictFreeSets}{
					\If{attack[0] $\in$ set AND attack[1] $\in$ set}{
						\nl \textit{conflictFreeSets} += \textit{set} - \textit{\{attack[0]\}} \;
						\nl \textit{conflictFreeSets} += \textit{set} - \textit{\{attack[1]\}} \;
						\nl \textit{conflictFreeSets} -= \textit{set} \;
					}
				}
			}
		\Else{
			\nl \textit{conflictFreeSets} += \textit{args} - \textit{\{attack[0]\}} \;
			\nl \textit{conflictFreeSets} += \textit{args} - \textit{\{attack[1]\}} \;
		}
	}
	\label{alg:mcf}
\end{algorithm}

\subsubsection{Sets}

The algorithm \ref{alg:mcf} is the representation of calculation of the Maximal Conflict Free sets. Depending on the size of the Argumentation Framework and it's complexity this approach will depend on the creation of the large number of combinations of sets. Furthermore, since the size of the collection of the conflict free sets can be $2^n$, where \textit{n} is the total number of arguments within the framework, iterating through this collection for each attack can be time consuming. Hence, to reduce the cost of running the above algorithm, first implementation involved sets data structures. "Set object is an unordered collection of distinct hashable objects" \citep{python_sets}.

To implement the algorithm the conflict free sets were stored as frozenset inside a set. Since the objects within the set are hashed, it provides fast access to its elements. Furthermore, testing membership within the set is performed in the constant time as it does not need to iterate through the whole set as it is the case with list data structure as it can be seen in table \ref{table:timingsMembership}. Additional benefit of using set is its property of being a collection of distinct objects. This helps to ensure that the duplicate sets are not stored, reducing the time required to iterate through the frozensets.

\begin{table}[h]
	\centering
	\caption{Timing in sec for testing membership (10,000 repetitions)}
	\label{table:timingsMembership}
	\begin{tabular}{lllll}
		\hline
		Number of members & 10    & 100   & 1000   & 10000 \\ \hline
		Set                & 0.011 & 0.038 & 0.472  & 4.898 \\
		List               & 0.015 & 0.558 & 54.401 & 57248.254 \\     
	\end{tabular}
\end{table}

Although the set data structure is superior to list in terms of performance, it has a massive overhead in terms of memory usage. Since its objects are hashed it is memory intensive especially for large collections. As can be seen in table \ref{table:sizeDataStructures}, for 10000 elements, the size of the set is over five times of the list. Since the implementation used set of frozensets, the required memory was growing exponentially for bigger argumentation frameworks, making the solution unusable.

\begin{table}[h]	
	\centering
	\caption{Size of data structures in bytes}
	\label{table:sizeDataStructures}
	\begin{tabular}{lllll}
		\hline
		Number of elements & 10  & 100  & 1000  & 10000  \\ \hline
		Set                & 736 & 8416 & 32992 & 524512 \\
		List               & 200 & 1008 & 9112  & 90112 
	\end{tabular}
\end{table}

% add some results from alias

\subsubsection{PyTable}
In order to try resolve the memory issue caused by the large set of frozensets, attempt has been made to implement the solution for Maximal Conflict Free sets using PyTables. PyTables is a python module for "managing hierarchical datasets and designed to efficiently and easily cope with extremely large amounts of data" \citep{pytables}. It is using HDF5, which is a "data model, library and file format for storing and managing data" \citep{hdf5}. 

Using PyTables for storage of all generated sets will release the huge amount of memory required for using sets of sets. The algorithm used with PyTables is slightly modified version of the algorithm \ref{alg:mcf}. In order to maximize the performance while using PyTables, three tables are created:
\begin{enumerate}
	\item{Arguments Table - table for storing all arguments within the argumentation framework. The table consist of the name of the argument and its mapping}
	\item{Conflict Free Table - table for storing references to conflict free sets. This table consists of only one column - id}
	\item{Argument Conflict Free Bridge Table - this table is used for storing references of arguments within the conflict free sets. Consisting of three columns: argument id, conflict free set id and active flag, links the tables above.}
\end{enumerate}


\subsubsection{Dictionaries}
Another approach to implement the maximal conflict free set algorithm is to use data structure 'defaultdict'. 'Defaultdict' class provides a new dictionary-like object, where the first argument in the constructor specifies default factory attribute. It is a subclass of a standard 'dict' class in Python and majority of the methods are the same between 'defaultdict' and 'dict' classes \citep{defaultdict}.

The implementation uses two 'defaultdict' classes with sets as its default factory attribute. One of the dictionaries is used to store all the arguments from the argumentation framework as the keys and a set of references to the conflict free sets as a value. Second dictionary is used to store the conflict free sets, where the key is just the identifier that gets incremented with every new record and the value is a set of the argument names in that particular conflict free set. Table \ref{table:dictsRepresentation} shows an example of the dictionaries being used to compute the maximal conflict free sets based on the argumentation framework \ref{fig:argumentationFrameworkFigure} from section \ref{abstractArgumentation}.

\begin{table}[h]
	\centering
	\caption{Representation of two dictionaries and their relations for Maximal Conflict Free sets generation}
	\label{table:dictsRepresentation}
	\begin{tabular}{lllll}
		\multicolumn{2}{l}{Arguments}                            &                       & \multicolumn{2}{l}{Conflict Free Sets}                  \\ \cline{1-2} \cline{4-5} 
		\multicolumn{1}{|l|}{Key} & \multicolumn{1}{l|}{Value}   & \multicolumn{1}{l|}{} & \multicolumn{1}{l|}{Key} & \multicolumn{1}{l|}{Value}   \\ \cline{1-2} \cline{4-5} 
		\multicolumn{1}{|l|}{a}   & \multicolumn{1}{l|}{\{1,2\}} & \multicolumn{1}{l|}{} & \multicolumn{1}{l|}{1}   & \multicolumn{1}{l|}{\{a\}}   \\ \cline{1-2} \cline{4-5} 
		\multicolumn{1}{|l|}{b}   & \multicolumn{1}{l|}{\{3,4\}} & \multicolumn{1}{l|}{} & \multicolumn{1}{l|}{2}   & \multicolumn{1}{l|}{\{a,c\}} \\ \cline{1-2} \cline{4-5} 
		\multicolumn{1}{|l|}{c}   & \multicolumn{1}{l|}{\{2\}}   & \multicolumn{1}{l|}{} & \multicolumn{1}{l|}{3}   & \multicolumn{1}{l|}{\{b\}}   \\ \cline{1-2} \cline{4-5} 
		\multicolumn{1}{|l|}{d}   & \multicolumn{1}{l|}{\{4\}}   & \multicolumn{1}{l|}{} & \multicolumn{1}{l|}{4}   & \multicolumn{1}{l|}{\{b,d\}} \\ \cline{1-2} \cline{4-5} 
	\end{tabular}
\end{table}

As can be seen in table \ref{table:dictsRepresentation}, there is a direct link between the arguments and conflict free sets dictionaries. The benefit of using a dictionaries as a data structure is that the retrieval of keys are a very fast operation, as those are usually implemented as a hash tables \citep{pythonDictionary}. Hence, the implementation is operating on two dictionaries simultaneously in order to provide the quick retrieval of the values based on the keys, while keeping the memory usage at a low level compared to using sets of sets.

% comparison of memory usage of defaultdict

There are number of steps that need to be taken in order to compute the maximal conflict free set using the dictionaries. Firstly, the sets to be modified need to be found, based on the attack tuple being added - this is simply intersection of the sets from arguments dictionary for the attacking and attacked argument. Then for all the values from that intersection, access the sets in conflict free sets dictionary and replace the existing set with two copies of that set: one without the attacker and one without the attacking arguments. The final step is to update the arguments dictionary by updating the references to the conflict free sets for all the arguments that were included in that set.

% some performance chart?

The operations in the approach above consists of accessing the values from the dictionary and performing simple operations on sets. Those operations are performed relatively quickly. However, similarly to other approaches, the number of sets in the conflict free sets dictionary in the worst case scenario will exponentially grow with each attack being added. If the argumentation framework is linear, i.e. a attacks b, b attacks c, c attacks d, etc., then with each attack that is processed the number of conflict free sets and number of operations will double, ultimately giving $O(n^2)$ big O notation in the worst case scenario. 



