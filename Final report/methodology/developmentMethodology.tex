\section{Software Development Methodology}
Software development methodology refers to the practice of dividing the project into individual stages and management of tasks required for the completion of the project. There are number of methodologies being used and most famous are Waterfall, Agile, Rapid Application Development (RAD) or DevOps. 

\subsection{Project Development Methodology}
Due to the complexity of the project and in order to factor the unknowns, best development methodology for this project is agile approach. The important characteristics of the agile methodologies are the short iterative cycles, where at the end of each one, a working product is delivered. Short cycles allows for a rapid feedback and will encapsulate all steps required to complete features selected for given iteration \citep{agile1}.

One of the drives for applying agile methodology to this project were expanding requirements. Not all the requirements have been defined at the beginning of the project. Some of the features, like ALIAS Web Extension, have been introduced at the later stages of the project. Hence, the flexible and adaptive methodology like agile was essential. 

Another important aspect of agile is rapid feedback. At the end of each iteration, feedback is provided in terms of delivered product \citep{agilebook}. In terms of this project, the feedback consisted of the test results of the implemented solution, providing insight into any performance and scalability issues. This in turn allowed for any required changes to be applied in the next iteration. Furthermore, regular feedback from project supervisor, Dr Simon Wells, helped to select the critical functionality of ALIAS and shape the final solution.

Although agile methodology is best applied to small teams, the principles of prioritizing tasks and workload, and short iteration cycles can be applied to projects with single developer. That will be the case with this project. 

\subsection{Project Schedule}

\begin{landscape}
	\centering
	\begin{figure}[]
		\includegraphics[width=21cm]{gantt}
		\caption{Project Schedule}
		\label{fig:projectSchedule}
	\end{figure}
\end{landscape}

This project can be divided into 4 stages:
\begin{enumerate}
	\item Research - at this stage, argumentation theory and semantics of abstract argumentation frameworks are researched. Furthermore, different approaches for computing the semantics are discovered and reviewed. Although majority of the research would be done at the beginning of the project, additional research might be required at later stages as well
	\item Writing - this stage consist of writing the findings of the research, summarizing the design, implementation and evaluation
	\item Development - at this stage, proposed solutions for computing abstract argumentation will be implemented. 
	\item Benchmark Testing - this stage consist of performing benchmark testing of all of the implemented solutions
\end{enumerate}

Figure \ref{fig:projectSchedule} shows the schedule of the whole project throughout the year. Only high-level tasks are included in the provided Gantt chart, such as research, write literature review, etc. However, those tasks have been further split into smaller, more manageable tasks.  


As can bee seen in figure \ref{fig:projectSchedule}, there were 4 development cycles planned. Each development cycle is 3 weeks long and consist of planning session, where number of features to be developed are selected for given iterations, design and development of the solution and testing to verify the application behaves correctly. During the project, development cycles have also been used to investigate and implement different approaches to computing abstract argumentation semantics.
