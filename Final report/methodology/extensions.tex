\section{Extensions}
ALIAS has been implemented to provide all admissible sets from the SAT solver. Those sets are then verified if they are part of the requested extension. 


\subsection{Stable Extension} \label{section:stableExtension}
Based on the definition \ref{def:stableExtension}, Stable Extension is the conflict-free set that attacks all arguments not included in the set. This makes verifying the Stable Extension a straightforward process using just a set theory: attacks of the set to be verified has to be equal to the set of arguments not included in the set to be checked. If those are equal, then the set in question is a stable extension, since there are no other arguments that are neither attacked by this set nor included in it. Hence, ALIAS verifies all the solutions provided by the SAT solver using the above property of the extension. This can be represented as: 

\begin{equation}
R^+(S) = args \oplus S
\end{equation}

where $S$ is the set of arguments to be verified, $R^+(S)$ represents the arguments attacked by the set $S$, and $\oplus$ indicates the symmetric difference. If the above equation evaluates to \textit{true}, the set being checked is added to the list of extensions, otherwise, it is ignored. Once all the sets have been checked, the generated list is returned.

\subsection{Matrices}
Matrix in mathematics is an 2 dimensional array with predefined number \textit{n} of rows and \textit{m} of columns: \textit{n x m} matrix \citep{matrices}, where each individual cell within the matrix stores information. Matrix with equal number of rows and columns is called a square matrix. 

As it was shown in the definition \ref{AFdef}, abstract argumentation is a collection of finite arguments and relations between those arguments. Hence, argumentation frameworks can be easily represented in the form of matrices, where both rows and columns represent the arguments within the framework and their relations are marked in each individual cells. 

If we consider the argumentation framework from figure \ref{fig:argumentationFrameworkFigure}, it can be seen that it consists of four arguments: a, b, c and d. Furthermore, it can be observed that a is attacking b, b is attacking c and c is also attacked by d. Hence, the argumentation framework can be represented as a 4 x 4 matrix shown in figure \ref{fig:matrixRepresentation}.

\begin{figure}[h]
\centering
	\begin{blockarray}{ccccc}
		  & a & b & c & d\\
		\begin{block}{c[cccc]}
			a & 0 & 1 & 0 & 0 \\
			b & 0 & 0 & 1 & 0 \\
			c & 0 & 0 & 0 & 0 \\
			d & 0 & 0 & 1 & 0 \\
		\end{block}	
	\end{blockarray}
	\label{fig:matrixRepresentation}
	\caption{Matrix representation of Argumentation Framework from figure \ref{fig:argumentationFrameworkFigure}}
\end{figure}

The only values used within the matrix are 0 and 1, where 1 indicates that there is a relation between the arguments from the relevant row and column. Furthermore, the direction of the attack can be easily described in the matrix, where each row represents a single argument, and the columns are used to show any relations to other arguments directed from that particular argument. Hence, when argument \textit{a} attacks argument \textit{b}, cell $M_{ab}$ has value 1, as shown in figure \ref{fig:matrixRepresentation}. In case the argument is a self attacking, the matrix value for a cell \textit{i, j}, where \textit{i} = \textit{j}, would be set to 1 \citep{afmatrices1}.

As \citet{afmatrices1} points out, matrices can only be used to answer a single question with reasonable efficiency from the work of \citet{bench2007argumentation}: 'Is \textit{A} an extension?'. And although \citet{afmatrices1} implemented a system called ASSA \citep{assa}, to compute all stable extensions using purely matrices, this approach is inefficient. In order to find all stable extensions the proposed solver needs to convert all possible instances of the selected set of arguments into a vector form. It then combines all vectors into a single massive matrix \citep{afmatrices1}. Thus, the proposed approach is ineffective, especially when compared to other available solvers.

\citet{matrix2} in their paper presented a different approach of using the matrices to verify if the given set is an extension. The argumentation framework can be divided into three parts: 
\begin{enumerate}
	\item The conflict-free set $S$
	\item The attacked set: $R^+(S)$
	\item The remaining set: $A\setminus (S\cup R^+(S))$
\end{enumerate}

By using sub-blocks of the matrix for different parts of the argumentation framework and transforming the matrix into one of the two standard forms, the proposed approach allows checking if the given set is conflict-free, admissible, stable, complete, preferred and grounded extensions \citep{matrix2}.

This approach has been used in ALIAS to verify complete and preferred extensions.

\subsection{Complete Extension}
The implementation for verifying if the given set is a Complete Extension is based on the approach presented by \citet{matrix2}. By dividing the matrix into certain sub-blocks, they found a dependency between column and row vectors that allow confirming if the set is, in fact, a complete extension.

Lets consider an example presented by \citet{matrix2}: given the argumentation framework $AF = (Args, attacks)$, where $Args = \{1,2,3,4,5\}$ and $attacks = \{(2,5), (3,4), (4,3), (5,1), (5,3)\}$, it can be represented in a graph as shown in figure \ref{fig:exampleAF}.

\begin{figure}[h]
	\tikzset{
		main/.style={draw, rectangle, rounded corners, minimum height=1cm, minimum width=4.5cm},
		arrow/.style={thick,<-,>=stealth}
	}
	\centering
	\begin{tikzpicture}[auto,node distance=1.5cm]
	\coordinate(coor);
	\node[draw=none,fill=none][above=0.75cm of coor](2){2};
	\node[draw=none,fill=none][below=0.75cm of coor](1){1};
	\node[draw=none,fill=none][right=of coor](5){5};
	\node[draw=none,fill=none][right=of 5](3){3};
	\node[draw=none,fill=none][right=of 3](4){4};			
	%%% ARROWS %%%
	\draw[arrow](5) -- (2);
	\draw[arrow](1) -- (5);
	\draw[arrow](3) -- (5);
	\draw[thick,<-,>=stealth,transform canvas={yshift=-0.2em}](3) -- (4);
	\draw[thick,<-,>=stealth,transform canvas={yshift=0.5em}](4) -- (3);
	\end{tikzpicture}
	\caption{Argumentation Framework \ref{fig:exampleAF}}
	\label{fig:exampleAF}
\end{figure}

The above argumentation framework can also be represented in form of the matrix as shown in figure \ref{fig:exampleAFMatrix}.

\begin{figure}[h]
	\centering
	\begin{blockarray}{cccccc}
		& 1 & 2 & 3 & 4 & 5\\
		\begin{block}{c[ccccc]}
			1 & 0 & 0 & 0 & 0 & 0 \\
			2 & 0 & 0 & 0 & 0 & 1 \\
			3 & 0 & 0 & 0 & 1 & 0 \\
			4 & 0 & 0 & 1 & 0 & 0 \\
			5 & 1 & 0 & 1 & 0 & 0 \\
		\end{block}	
	\end{blockarray}
	\caption{Matrix representation of Argumentation Framework from figure \ref{fig:exampleAF}}
	\label{fig:exampleAFMatrix}
\end{figure}

Lets consider an admissible set $S = \{1,2\}$. In order to verify if the set $S$ is complete, the matrix should be divided into sub-blocks as shown in figure \ref{fig:subblocks}. The green area represents the set $S = \{1,2\}$, orange sub-block relation between the set $S$ and the remaining arguments indicated by $M^s(\{1,2\})$, while yellow sub-block of arguments not included in the set $S$ indicated by $M^c(\{1,2\})$. 

\citet{matrix2} propose the following theorem:
\begin{theorem}
Given $AF = (A, R)$ with $A = {1, 2, ..., n}$, an admissible extension $S = {i_1, i_2, ..., i_k} \subseteq A$ is complete if and only if:
\begin{enumerate}\label{theorem:completeMatrix}
	\item some column vector $M^s_{*,p}$ of the s-sub-block $M^s(S)$ is zero, then its corresponding column vector $M^c_{*,p}$ of the c-sub-block $M^c(S)$ is non-zero, and
	\item for each non-zero column vector $M^c_{*,p}$ of the c-sub-block $M^c(S)$ appearing in (1), there is at least one non-zero element $a_{j_q,j_p}$ of $M^c_{*,p}$ such that the corresponding column vector $M^s_{*,q}$ of the s-sub-block $M^s(S)$ is zero, where $\{j_1,j_2,...,j_h\} = A\backslash S$ and $1\leq q,p \leq h$.
\end{enumerate}
\end{theorem}


\begin{figure}[h]
	\centering
	\begin{tabular}{llllll}
		& \textbf{1} & \textbf{2} & \textbf{3} & \textbf{4} & \textbf{5} \\ \hline
		\multicolumn{1}{l|}{\textbf{1}} & \cellcolor[HTML]{32CB00}{\color[HTML]{FFFFFF} \textbf{0}} & \cellcolor[HTML]{32CB00}{\color[HTML]{FFFFFF} \textbf{0}} & \cellcolor[HTML]{F8A102}\textbf{0} & \cellcolor[HTML]{F8A102}\textbf{0} & \cellcolor[HTML]{F8A102}\textbf{0} \\
		\multicolumn{1}{l|}{\textbf{2}} & \cellcolor[HTML]{32CB00}{\color[HTML]{FFFFFF} \textbf{0}} & \cellcolor[HTML]{32CB00}{\color[HTML]{FFFFFF} \textbf{0}} & \cellcolor[HTML]{F8A102}\textbf{0} & \cellcolor[HTML]{F8A102}\textbf{0} & \cellcolor[HTML]{F8A102}\textbf{1} \\
		\multicolumn{1}{l|}{\textbf{3}} & 0 & 0 & \cellcolor[HTML]{FCFF2F}{\color[HTML]{000000} \textbf{0}} & \cellcolor[HTML]{FCFF2F}{\color[HTML]{000000} \textbf{1}} & \cellcolor[HTML]{FCFF2F}{\color[HTML]{000000} \textbf{0}} \\
		\multicolumn{1}{l|}{\textbf{4}} & 0 & 0 & \cellcolor[HTML]{FCFF2F}{\color[HTML]{000000} \textbf{1}} & \cellcolor[HTML]{FCFF2F}{\color[HTML]{000000} \textbf{0}} & \cellcolor[HTML]{FCFF2F}{\color[HTML]{000000} \textbf{0}} \\
		\multicolumn{1}{l|}{\textbf{5}} & 1 & 0 & \cellcolor[HTML]{FCFF2F}{\color[HTML]{000000} \textbf{1}} & \cellcolor[HTML]{FCFF2F}{\color[HTML]{000000} \textbf{0}} & \cellcolor[HTML]{FCFF2F}{\color[HTML]{000000} \textbf{0}}
	\end{tabular}
	\caption{Matrix sub-blocks of Argumentation Framework \ref{fig:exampleAF} for set $S = \{1,2)\}$}
	\label{fig:subblocks}
\end{figure}

From the figure \ref{fig:subblocks}, it can be seen that the $M^s(\{1,2\})$, contains two zero column vectors: $M^s_{*,1} = \begin{pmatrix}
0\\ 
0
\end{pmatrix}$ and $M^s_{*,2} = \begin{pmatrix}
0\\ 
0
\end{pmatrix}$ . Their corresponding column vectors in $M^c(\{1,2\})$ are $M^c_{*,1} = \begin{pmatrix}
0\\ 
1\\ 
1
\end{pmatrix}$ and $M^c_{*,2} = \begin{pmatrix}
1\\ 
0\\ 
0
\end{pmatrix}$, which are all non-zero. Furthermore, for $a_{43} = 1$ in $M^c_{*,1}$, the corresponding column vector $M^s_{*,2}$ in $M^s(\{1,2\})$ is zero. And for $a_{34} = 1$ in $M^c(\{1,2\})$, the corresponding column vector $M^s_{*,1}$ in $M^s(\{1,2\})$ is also zero. Hence based on the theorem \ref{theorem:completeMatrix}, subset $S = \{1,2\}$ is a Complete Extension of $AF$.

In order to implement this in Python, SciPy \citep{jones2014scipy} sparse matrix object has been used. Class Matrix has been created that takes the responsibility of matrix creation and accessing the required sub-blocks of the matrix. Once the solver provides a possible solution, then both C and S sub-blocks of the matrix are fetched. Once all the required data is gathered, the first condition of the theorem \ref{theorem:completeMatrix} can be verified. The columns of each sub-block are summed, zipped together and compared. If the sum of the corresponding columns in s-block and c-block are equal to 0 then the application breaks out of the method returning False. 

Once all the column sums have been checked the application can move to the second condition of the theorem. Search is performed to filter out all the elements from c sub-block with value 1. For each element $a_{i,j}$ from the generated list, the corresponding value from the sum of columns of s sub-block is checked. The dictionary holding results for all columns is updated setting the value to either True if the corresponding column vector from s sub-block is 0, or False otherwise. There are additional checks to ensure the record does not already exist in the dictionary, as this condition has to be met for at least one non-zero element from each column vector of c sub-block.

After checking all elements from C and S sub-blocks, the dictionary storing results of the second conditions is checked. If any of the elements of the dictionary is set to false, the method returns false and the set is not considered as part of the complete extension. 

\subsection{Preferred Extension} \label{section:preferredExtension}
Preferred extensions are checked in a similar way to the Complete Extensions above. By definition \ref{def:preferredExtension}, Preferred Extension is the maximal, with respect to set inclusion, an admissible set of arguments. Hence, the only checks we have to do is to verify if the given set is admissible and if it is maximal with respect to set inclusion.

Similarly to the complete extension, SciPy matrix representations have been used for the argumentation frameworks. Once the solver generates a possible solution, the fetch of arguments to be checked is performed. However, in the case of the Preferred extension, the application only fetches the arguments not included in the proposed solution and not attacked by the solution. Hence, the required matrix is a c sub-block without the attacked set $R^+(S)$.
 
Once the solver fetches required sub-block it sums all the column vertices. This creates a single list of sums for all arguments not being attacked by the set $S$. Finally, the algorithm iterates through that list to check if any value is equal to 0. If it is, then the set of arguments being currently checked is not a preferred extension. Otherwise, the method will return the true and that set of arguments will be added to the list of preferred extensions.