\subsection{Extensions}
\subsubsection{Stable Extension}
Based on the definition \ref{def:stableExtension}, Stable Extension is the conflict-free set that attacks all arguments not included in the set. Hence, it can be said that the Stable Extension are the Maximal Conflict Free sets of the argumentation framework. This makes verifying the Stable Extension a straight forward process using just a set theory: attacks of the set to be verified has to be equal to the set of arguments not included in the set to be checked. If those are equal, then the set in question is a stable extension, since there are no other arguments that are neither attacked by this set nor included in it. Hence, Alias verifies all the solutions provided from the SAT solver using above property of the extension. This can be represented as: 

\begin{equation}
R^+(S) = args \oplus S
\end{equation}

where $S$ is the set of arguments to be verified, $R^+(S)$ represents the arguments attacked by the set $S$, and $\oplus$ indicates the symmetric difference. If the above equation evaluates to \textit{true}, the give

\subsubsection{Matrices}
Matrix in mathematics is an 2 dimensional array with predefined number \textit{n} of rows and \textit{m} of columns: \textit{n x m} matrix \citep{matrices}, where each individual cell within the matrix stores information. Matrix with equal number of rows and columns is called a square matrix. 

As it was shown in the definition \ref{AFdef}, abstract argumentation is a collection of finite arguments and relations between those arguments. Hence, argumentation frameworks can be easily represented in the form of matrices, where both rows and columns represent the arguments within the framework and their relations are marked in each individual cells. 

If we consider argumentation framework from figure \ref{fig:argumentationFrameworkFigure}, it can be seen that it consists of four arguments: a, b, c and d. Furthermore, it can be observed that a is attacking b, b is attacking c and c is also attacked by d. Hence, the argumentation framework can be represented as a 4 x 4 matrix:

\begin{figure}[h]
\centering
	\begin{blockarray}{ccccc}
		  & a & b & c & d\\
		\begin{block}{c[cccc]}
			a & 0 & 1 & 0 & 0 \\
			b & 0 & 0 & 1 & 0 \\
			c & 0 & 0 & 0 & 0 \\
			d & 0 & 0 & 1 & 0 \\
		\end{block}	
	\end{blockarray}
	\label{fig:matrixRepresentation}
	\caption{Matrix representation of Argumentation Framework from figure \ref{fig:argumentationFrameworkFigure}}
\end{figure}

The only values used within the matrix are 0 and 1, where 1 indicates that there is a relation between the arguments from relevant row and column. Furthermore, the direction of the attack can be easily described in the matrix, where each row represents a single argument, and the columns are used to show any relations to other arguments directed from that particular argument. In case the argument is a self attacking argument, the matrix value for a cell \textit{i, j}, where \textit{i} = \textit{j}, would be set to 1 \citep{afmatrices1}.

As \citet{afmatrices1} points out, matrices can only be used to answer a single question with reasonable efficiency from the work of \citet{bench2007argumentation}: 'Is \textit{A} an extension?'. And although \citet{afmatrices1} implemented system called ASSA \citep{assa}, to compute all stable extensions using purely matrices, this approach is inefficient. In order to find all stable extensions the proposed solver needs to create all possible instances of selected set of arguments into a vector form. It then combines all vectors into a single massive matrix \citep{afmatrices1}. Thus, the proposed approach is ineffective, especially when compared to other available solvers.

\citet{matrix2} in his paper presented different approach of using the matrices to verify if the given set is an extension. The argumentation framework can be divided into three parts: 
\begin{enumerate}
	\item The conflict-free set $S$
	\item The attacked set: $R^+(S)$
	\item The remaining set: $A\setminus (S\cup R^+(S))$
\end{enumerate}

By using sub-blocks of the matrix for different parts of the argumentation framework and transforming the matrix into one of the two standard forms, the proposed approach allows to check if the given set is conflict free, admissible, stable, complete, preferred and grounded extensions \citep{matrix2}.

\subsubsection{Preferred Extension}
\subsubsection{Complete Extension}