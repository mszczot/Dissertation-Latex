%%You can delete all the comments after you have finished your document
%this sets up the defaults for the documents, 12pt font and A4 size. The article type sets this up as such as opposed to letter or memo.

%for the finer points LaTeX see https://en.wikibooks.org/wiki/LaTeX or http://tex.stackexchange.com/

\documentclass[12pt,a4paper]{article}
\usepackage{titlesec} %these are how we import packages, one helps set up footers and title layout
\usepackage{fancyhdr}

% !TEX TS-program = pdflatex
% !TEX encoding = UTF-8 Unicode
\usepackage[utf8]{inputenc} % set input encoding (not needed with XeLaTeX)
\usepackage{graphicx} % support the \includegraphics command and options

% \usepackage[parfill]{parskip} % Activate to begin paragraphs with an empty line rather than an indent

%%% PACKAGES
\usepackage{booktabs} % for much better looking tables
\usepackage{array} % for better arrays (eg matrices) in maths
\usepackage{paralist} % very flexible & customisable lists (eg. enumerate/itemize, etc.)
\usepackage{verbatim} % adds environment for commenting out blocks of text & for better verbatim
\usepackage{subfig} % make it possible to include more than one captioned figure/table in a single float
\usepackage[toc,page]{appendix}
\usepackage{amsthm}
\usepackage[square,sort]{natbib}
\usepackage{tikz}
\usetikzlibrary{er,positioning}
\usepackage{amsmath}
\usepackage{supertabular}
\usepackage{pdfpages}
\usepackage{enumitem}
\usepackage{longtable}
\usepackage{rotating}
\usepackage{pdflscape}
\usepackage{pdfpages}
\usepackage{verbatim}
\usepackage{pgfplots}

\graphicspath{{./img/}}

\theoremstyle{definition}
\newtheorem{definition}{Definition}[section]
% These packages are all incorporated in the memoir class to one degree or another...

%header and footer settings
\pagestyle{fancyplain}
\fancyhf{}
\renewcommand{\headrulewidth}{0.5pt}
\renewcommand{\footrulewidth}{0.5pt}
\setlength{\headheight}{15pt}
\fancyhead[L]{Marcin Szczot - 40180425}
\fancyhead[R]{ SOC10101 Honours Project}
\fancyfoot[L]{}
\fancyfoot[C]{\thepage}

%set better section layout
\makeatletter
\renewcommand\subsection{\@startsection {subsection}{1}{2mm} % name, level, indent
                               {3pt plus 2pt minus 1pt} % before skip
                               {3pt plus 0pt} % after skip
                               {\normalfont\bfseries}}
\renewcommand\subsubsection{\@startsection {subsubsection}{2}{4mm} % name, level, indent
                               {3pt plus 2pt minus 1pt} % before skip
                               {3pt plus 0pt} % after skip
                               {\normalfont\bfseries}}
\makeatother
\makeatletter
\renewcommand\section{\@startsection {section}{1}{0mm} % name, level, indent
                               {4pt plus 2pt minus 1pt} % before skip
                               {4pt plus 0pt} % after skip
                               {\bfseries}}
\makeatother

\pgfplotstableread[row sep=\\,col sep=&]{
    Solver & Time & Score\\
pyglaf & 222.2444 & 1226 \\
cegartix & 263.6557 & 1181 \\
argmat-sat & 245.8243 & 1167\\
argmat-dvisat & 291.0360 & 1147\\
CoQuiAAS & 186.4464 & 1134\\
argmat-mpg & 316.5906 & 1125\\
goDIAMOND & 224.2296 & 1110\\
heureka & 	272.8588 & 1021\\
conarg & 	184.4099 & 1017\\
ArgTools & 553.4805 & 935\\
ArgSemSAT & 388.0886 & 905\\
EqArgSolver & 142.1794 & 422\\
argmat-clpb & 1270.5737 & 40\\
gg-sts & 	149.6766 & -1160\\
    }\mydata



%this starts the document
\begin{document}

%you can import other documents into your main one, these layout the Title and Declarations on its own page.
%you might need to change these to \ if your on Microsoft Windows.
\input{./Dissertation-Title.tex}
\input{./Dissertation-Dec.tex}
\pagebreak
\input{./Dissertation-DP.tex}
\pagebreak

%LaTeX let you define the abstract separately so it wont get sucked into the main document.
\begin{abstract}
\documentclass[../Dissertation.tex]{subfiles}

\begin{document}
	
\end{document}
\end{abstract}
\pagebreak

\tableofcontents % is generated for you
\newpage

\listoftables
%generated in same way as figures
\newpage

\listoffigures
%you may have captions such as equations, listings etc they should all appear as required
%these are done for you as long as you use \begin{figure}[placement settings] .. bla bla ... \end{figure}
\newpage

\section*{Acknowledgements}
Insert acknowledgements here
\subsection*{}
	I would like to thank my cat, dog and family.
\newpage

%-------------------------------------------------------------------------------
% Main content 
%-------------------------------------------------------------------------------
\section{Introduction}
\documentclass[../Dissertation.tex]{subfiles}

\begin{document}
	This is my introduction for my project
\end{document}

\section{Literature Review}
\documentclass[../Dissertation.tex]{subfiles}

\begin{document}
	\subsection{Abstract Argumentation Semantics}
	This part explains the different types of semantics.
	
	\subsection{Existing solutions}
	\subsubsection{pyglaf}
	bla bla bla
\end{document} 

%-------------------------------------------------------------------------------
% End of Main content 
%-------------------------------------------------------------------------------

\newpage
\bibliographystyle{apalike}
\bibliography{references}
%example of References. See https://en.wikibooks.org/wiki/LaTeX/Bibliography_Management
%might be good to use a separate document for these so your main work is not one really long text file. 

%you can crate this on a extra tex document just like the title or any other part of the document.
\newpage
\begin{appendices}
\section{Initial Project Overview}
\label{appendix:IPO}

\includepdf[pages={1-},scale=1]{./diaries/IPO.pdf}

\section{Requirement Analysis} 
\label{appendix:requirementAnalysis}
\begin{center}
\begin{longtable}{| p{.02\textwidth} | p{.80\textwidth} | p{.18\textwidth} |} 
\caption{Software Requirements}
\label{tab:requirementsAnalysis}\\
\hline \multicolumn{1}{|c|}{\textbf{ID}} & \multicolumn{1}{c|}{\textbf{Requirement}} & \multicolumn{1}{c|}{\textbf{MoSCoW}} \\ \hline 
\endfirsthead


\multicolumn{3}{c}%
{{\bfseries \tablename\ \thetable{} -- continued from previous page}} \\
\hline \multicolumn{1}{|c|}{\textbf{ID}} &
\multicolumn{1}{c|}{\textbf{Requirement}} &
\multicolumn{1}{c|}{\textbf{MoSCoW}} \\ \hline 
\endhead

\hline \multicolumn{3}{|r|}{{Continued on next page}} \\ \hline
\endfoot

\hline \hline
\endlastfoot

1  & Should read provided tgf file                                                                                                                           & Must have   \\ \hline
2  & Should parse tgf file                                                                                                                                   & Must have   \\ \hline
3  & Should read provided apx file                                                                                                                           & Should have \\ \hline
4  & Should be able to determine SOME complete extensions from given argumentation framework (SE-CO)                                                         & Should have \\ \hline
5  & Should be able to determine ALL complete extensions from given argumentation framework (EE-CO)                                                          & Must have \\ \hline
6  & Given the argumentation framework and some argument, should decide whether the given argument is credulously inferred in complete extension (DC-CO)     & Should have \\ \hline
7  & Given the argumentation framework and some argument, should decide whether the given argument is sceptically inferred in complete extension (DS-CO)     & Should have \\ \hline
8  & Should be able to determine SOME preferred semantics from given argumentation framework (SE-PR)                                                         & Should have \\ \hline
9  & Should be able to determine ALL preferred extensions from given argumentation framework (EE-PR)                                                         & Must have \\ \hline
10 & Given the argumentation framework and some argument, should decide whether the given argument is credulously inferred in preferred extension (DC-PR)    & Should have \\ \hline
11 & Given the argumentation framework and some argument, should decide whether the given argument is sceptically inferred in preferred extension (DS-PR)    & Should have \\ \hline
12 & Should be able to determine SOME stable extensions from given argumentation framework (SE-ST)                                                           & Should have \\ \hline
13 & Should be able to determine ALL stable extensions from given argumentation framework (EE-ST)                                                            & Must have \\ \hline
14 & Given the argumentation framework and some argument, should decide whether the given argument is credulously inferred in stable extension (DC-ST)       & Should have \\ \hline
15 & Given the argumentation framework and some argument, should decide whether the given argument is sceptically inferred in stable extension (DS-ST)       & Should have \\ \hline
16 & Should be able to determine SOME semi-stable extensions from given argumentation framework (SE-SST)                                                     & Should have \\ \hline
17 & Should be able to determine ALL semi-stable extensions from given argumentation framework (EE-SST)                                                      & Should have \\ \hline
18 & Given the argumentation framework and some argument, should decide whether the given argument is credulously inferred in semi-stable extension (DC-SST) & Should have \\ \hline
19 & Given the argumentation framework and some argument, should decide whether the given argument is sceptically inferred in semi-stable extension (DS-SST) & Should have \\ \hline
20 & Should be able to determine SOME stage extensions from given argumentation framework (SE-STG)                                                           & Should have \\ \hline
21 & Should be able to determine ALL stage extensions from given argumentation framework (EE-STG)                                                            & Should have \\ \hline
22 & Given the argumentation framework and some argument, should decide whether the given argument is credulously inferred in stage extension (DC-STG)       & Should have \\ \hline
23 & Given the argumentation framework and some argument, should decide whether the given argument is sceptically inferred in stage extension (DS-STG)       & Should have \\ \hline
24 & Should be able to determine SOME grounded extensions from given argumentation framework (SE-GR)                                                         & Should have \\ \hline
25 & Should be able to determine ALL grounded extensions from given argumentation framework (EE-GR)                                                          & Should have \\ \hline
26 & Should be able to determine SOME ideal extensions from given argumentation framework (SE-ID)                                                            & Should have \\ \hline
27 & Should be able to determine ALL ideal extensions from given argumentation framework (SE-ID)                                                             & Should have \\ \hline
28 & Should output the result of the computation to command line                                                                                             & Must have   \\ \hline
29 & Should take file format as a command line argument, i.e. -fo {[}file format{]}                                                                          & Must have   \\ \hline
30 & Should take type of task as a command line argument, i.e. -p {[}EE-CO{]}                                                                                & Must have   \\ \hline
31 & Should take file as a command line argument, i.e. -f {[}./argumentationFramework.apx{]}                                                                 & Must have   \\ \hline 
\end{longtable}
\end{center}

\section{Project Schedule}
\label{appendix:projectSchedule}
\includegraphics[width=\textwidth]{gantt1}

\section{Meetings diaries}
\label{appendix:diaries}
\newpage

\includepdf{./diaries/20180130.pdf}
\includepdf{./diaries/20180206.pdf}

\section{ICCMA 2017 Submissions}
\label{appendix:ICCMASubmissions}
\begin{center}
\begin{longtable}{| p{.2\textwidth} | p{.8\textwidth} |}
\caption{ICCMA 2017 Submissions}
\label{table:ICCMA2017Submissions}\\

\hline \multicolumn{1}{|c|}{\textbf{Solver}} & \multicolumn{1}{c|}{\textbf{Author}}\\ \hline 
\endfirsthead


\multicolumn{2}{c}%
{{\bfseries \tablename\ \thetable{} -- continued from previous page}} \\
\hline \multicolumn{1}{|c|}{\textbf{Solver}} &
\multicolumn{1}{c|}{\textbf{Author}} \\ \hline 
\endhead

\hline \multicolumn{2}{|r|}{{Continued on next page}} \\ \hline
\endfoot

\hline \hline
\endlastfoot

argmat-clpb   & Fuan Pu (Tsinghua University, China), Guiming Luo (Tsinghua University, China), Yucheng Chen (Tsinghua University, China).                  \\ \midrule
argmat-dvisat & Fuan Pu (Tsinghua University, China), Guiming Luo (Tsinghua University, China), Ya Hang (Tsinghua University, China).                    \\ \hline
argmat-mpg & Fuan Pu (Tsinghua University, China), Guiming Luo (Tsinghua University, China), Ya Hang (Tsinghua University, China).                    \\ \hline
argmat-sat & Fuan Pu (Tsinghua University, China), Guiming Luo (Tsinghua University, China), Ya Hang (Tsinghua University, China).                    \\ \hline
ArgSemSAT  & Federico Cerutti (Cardiff University, UK), Mauro Vallati (University of Huddersfield, UK), Massimiliano Giacomin (University of Brescia, Italy), Tobia Zanetti (University of Brescia, Italy). \\ \hline
ArgTools   & Samer Nofal (German Jordanian University, Jordan), Katie Atkinson (University of Liverpool, UK), Paul E. Dunne (University of Liverpool, UK).              \\ \hline
ASPrMin    & Wolfgang Faber (University of Huddersfield, UK), Mauro Vallati (University of Huddersfield, UK), Federico Cerutti (Cardiff University, UK), Massimiliano Giacomin (University of Brescia, Italy). \\ \hline
cegartix   & Wolfgang Dvořák (TU Wien, Austria), Matti Järvisalo (University of Helsinki, Finland), Johannes P. Wallner (TU Wien, Austria).                 \\ \hline
Chimærarg  & Federico Cerutti (Cardiff University, UK), Mauro Vallati (University of Huddersfield, UK), Massimiliano Giacomin (University of Brescia, Italy).              \\ \hline
ConArg  & Stefano Bistarelli (Università di Perugia, Italy), Fabio Rossi (Università di Perugia, Italy), Francesco Santini (Università di Perugia, Italy).              \\ \hline
CoQuiAAS   & Jean-Marie Lagniez (Univ. Artois, France), Emmanuel Lonca (Univ. Artois, France), Jean-Guy Mailly (Univ. Paris Descartes, France).                \\ \hline
EqArgSolver   & Odinaldo Rodrigues (King's College London, UK).                                       \\ \hline
gg-sts  & Tomi Jahunen (Aalto University, Finland), Shahab Tasharrofi (Aalto University, Finland).                            \\ \hline
goDIAMOND  & Stefan Ellmauthaler (Leipzig University, Germany), Hannes Strass (Leipzig University, Germany).                           \\ \hline
heureka    & Nils Geilen (Universität Koblenz-Landau, Germany), Matthias Thimm (Universität Koblenz-Landau, Germany).                        \\ \hline
pyglaf  & Mario Alviano (University of Calabria, Italy).                                       
\end{longtable}
\end{center}

\begin{sidewaystable}
\caption{Tasks supported by solvers}
\label{table:tasksSupportedBySolvers}
\resizebox{\paperwidth}{!}{
\begin{tabular}{|p{.2\textwidth}|p{.032\textwidth}|p{.032\textwidth}|p{.032\textwidth}|p{.032\textwidth}|p{.032\textwidth}|p{.032\textwidth}|p{.032\textwidth}|p{.032\textwidth}|p{.032\textwidth}|p{.032\textwidth}|p{.032\textwidth}|p{.032\textwidth}|p{.032\textwidth}|p{.032\textwidth}|p{.032\textwidth}|p{.032\textwidth}|p{.032\textwidth}|p{.032\textwidth}|p{.032\textwidth}|p{.032\textwidth}|p{.032\textwidth}|p{.032\textwidth}|p{.032\textwidth}|p{.032\textwidth}|p{.032\textwidth}|}
\hline
     & D3 & \multicolumn{4}{l|}{CO} & \multicolumn{4}{l|}{PR} & \multicolumn{4}{l|}{ST} & \multicolumn{4}{l|}{SST} & \multicolumn{4}{l|}{STG} & \multicolumn{2}{l|}{GR} & \multicolumn{2}{l|}{ID} \\ \hline
     & & DC   & DS   & SE  & EE  & DC   & DS   & SE  & EE  & DC   & DS   & SE  & EE  & DC   & DS   & SE   & EE  & DC   & DS   & SE   & EE  & DC   & SE   & DC   & SE   \\ \hline
argmat-clpb   & & 1 & 1 & 1   & 1   &   &   &  &  & 1 & 1 & 1   & 1   &   &   &   &  &   &   &   &  & 1    & 1    &   &   \\ \hline
argmat-dvisat & 1  & 1 & 1 & 1   & 1   & 1 & 1 & 1   & 1   & 1 & 1 & 1   & 1   &   &   &   &  &   &   &   &  & 1    & 1    & 1    & 1    \\ \hline
argmat-mpg & 1  & 1 & 1 & 1   & 1   & 1 & 1 & 1   & 1   & 1 & 1 & 1   & 1   & 1 & 1 & 1 & 1   & 1 & 1 & 1 & 1   & 1    & 1    & 1    & 1    \\ \hline
argmat-sat & 1  & 1 & 1 & 1   & 1   & 1 & 1 & 1   & 1   & 1 & 1 & 1   & 1   & 1 & 1 & 1 & 1   & 1 & 1 & 1 & 1   & 1    & 1    & 1    & 1    \\ \hline
ArgSemSAT  & & 1 & 1 & 1   & 1   & 1 & 1 & 1   & 1   & 1 & 1 & 1   & 1   & 1 & 1 & 1 & 1   &   &   &   &  & 1    & 1    &   &   \\ \hline
ArgTools   & & 1 & 1 & 1   & 1   & 1 & 1 & 1   & 1   & 1 & 1 & 1   & 1   & 1 & 1 & 1 & 1   & 1 & 1 & 1 & 1   & 1    & 1    & 1    & 1    \\ \hline
ASPrMin    & &   &   &  &  &   &   &  & 1   &   &   &  &  &   &   &   &  &   &   &   &  &   &   &   &   \\ \hline
cegartix   & 1  & 1 & 1 & 1   & 1   & 1 & 1 & 1   & 1   & 1 & 1 & 1   & 1   & 1 & 1 & 1 & 1   & 1 & 1 & 1 & 1   & 1    & 1    & 1    & 1    \\ \hline
Chimærarg & &   &   &  &  &   &   &  & 1   &   &   &  & 1   &   &   &   &  &   &   &   &  &   &   &   &   \\ \hline
ConArg  & 1  & 1 & 1 & 1   & 1   & 1 & 1 & 1   & 1   & 1 & 1 & 1   & 1   & 1 & 1 & 1 & 1   & 1 & 1 & 1 & 1   & 1    & 1    & 1    & 1    \\ \hline
CoQuiAAS   & 1  & 1 & 1 & 1   & 1   & 1 & 1 & 1   & 1   & 1 & 1 & 1   & 1   & 1 & 1 & 1 & 1   & 1 & 1 & 1 & 1   & 1    & 1    & 1    & 1    \\ \hline
EqArgSolver   & 1  & 1 & 1 & 1   & 1   & 1 & 1 & 1   & 1   & 1 & 1 & 1   & 1   &   &   &   &  &   &   &   &  & 1    & 1    &   &   \\ \hline
gg-sts  & 1  & 1 & 1 & 1   & 1   & 1 & 1 & 1   & 1   & 1 & 1 & 1   & 1   & 1 & 1 & 1 & 1   & 1 & 1 & 1 & 1   & 1    & 1    & 1    & 1    \\ \hline
goDIAMOND  & 1  & 1 & 1 & 1   & 1   & 1 & 1 & 1   & 1   & 1 & 1 & 1   & 1   & 1 & 1 & 1 & 1   & 1 & 1 & 1 & 1   & 1    & 1    & 1    & 1    \\ \hline
heureka    & & 1 & 1 & 1   & 1   & 1 & 1 & 1   & 1   & 1 & 1 & 1   & 1   &   &   &   &  &   &   &   &  & 1    & 1    &   &   \\ \hline
pyglaf  & 1  & 1 & 1 & 1   & 1   & 1 & 1 & 1   & 1   & 1 & 1 & 1   & 1   & 1 & 1 & 1 & 1   & 1 & 1 & 1 & 1   & 1    & 1    & 1    & 1    \\ \hline
\end{tabular}
}
\end{sidewaystable}

\section{Argumentation Framework Graph}
\label{appendix:aficcma}
\includegraphics[width=0.9\textwidth]{AFICCMA}

%\begin{subappendices}
%\subsection{Example sub appendices}
...
%\end{subappendices}

%\section{Second Formal Review Output}
%Insert a copy of the project review form you were given at the end of the review by %the second marker

%\section{Diary Sheets (or other project management evidence)}
%Insert diary sheets here together with any project management plan you have

%\section{Appendix 4 and following}
%insert content here and for each of the other appendices, the title may be just on a page by itself, the pages of the appendices are not numbered, unless an included document such as a user manual or design document is itself pager numbered.
\end{appendices}

\end{document}
