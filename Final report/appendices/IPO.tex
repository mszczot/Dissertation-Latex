\textbf{Initial Project Overview} \\
\textbf{SOC10101 Honours Project (40 Credits)}             \\                                          
\textbf{Title of Project: Computing abstract argumentation semantics }\\

\section{Overview of Project Content and Milestones}
The aim of this project is to implement scalable software for computing semantics (preferred, stable, grounded, etc.) of abstract argumentation from the given argumentation framework. This project will have a number of milestones that will need to be achieved. They include research of the argumentation framework semantics based on Dung’s original notions of complete, grounded, preferred and stable semantics, as well as further proposed notions of semi-stable, stage and ideal semantics. Furthermore, different approaches, such as labelling-based and extension-based approaches along with different algorithms will be reviewed, investigated and evaluated. Additionally, software submitted to the International Competition on Computational Models of Argumentation will be investigated and evaluated based on the approaches used for computing the semantics. Finally, new software will be designed, developed, tested and benchmarked against the solvers from International Competition on Computational Models of Argumentation. The aim will be to produced scalable software for calculating different types of semantics from provided argumentation framework.

\section{The Main Deliverable(s):}
The main deliverables of this project will be the final report, which will include the review of argumentation framework, common algorithms used for calculating abstract argumentation semantics and software to be submitted to ICCMA. The report will also include a review of design, development and testing of the proposed solution. Furthermore, the developed software for computing abstract argumentation semantics will be part of deliverables.

\section{The Target Audience for the Deliverable(s):}
The target audience for projects’ deliverables are the researches working on argumentation in computing.

\section{The Work to be Undertaken:}
This project requires research to be conducted on the argumentation framework and the notion of semantics introduced by Dung and extended by Caminada and others. Existing approaches and algorithms for labelling arguments and calculating the semantics will have to be researched and reviewed. Furthermore, new software for computing the semantics from given argumentation framework will be designed, implemented and tested. Existing solutions will be used for benchmarking the performance and scalability of the developed solution.

\section{Additional Information / Knowledge Required:}
In order to complete the project further knowledge is required in the abstract argumentation frameworks and their semantics. Furthermore, new system for computing the semantics will be developed in Python, hence extensive knowledge will be required for designing and implementing scalable approach for traversing large graphs of argumentation frameworks. 

\section{Information Sources that Provide a Context for the Project:}
The project will be based on the abstract argumentation semantics introduced by Phan Minh Dung in the paper On the acceptability of arguments and its fundamental role in nonmonotonic reasoning, logic programming and n-person games. The semantics were further extended with semi-stable semantics by Martin Caminada in his paper Semi-Stable Semantics, stage semantics by Bart Verheij in Two approaches to dialectical argumentation: admissible sets and argumentation stages and ideal semantics by Phan Minh Dung, Paolo Mancarella and Francesca Toni in Computing ideal sceptical argumentation. Furthermore, the solvers submitted to the International Competition on Computational Models of Argumentations (http://argumentationcompetition.org/) will be used for benchmarking the proposed solution. 

\section{The Importance of the Project:}
Research focus in AI for defeasible reasoning

\section{The Key Challenge(s) to be Overcome:}
Main challenge of this project will be to design and implement universal solution for computing different semantics (complete, preferred, stable, etc.) of abstract arguments in the provided argumentation framework. Furthermore, the solution must be able to scale appropriately and be able to perform calculation on supplied argumentation framework with different structures.
