Alias was being developed using iterative approach. For each iteration a different approach for computing the extensions has been implemented. In order to verify and benchmark the approach, each iteration had allocated time for testing. Results of all tests have been used to evaluate the final solution and provide the insight for development progress.

As described in section \ref{label:projectRequirements}, there are number of functional and non-functional requirements for Alias. One of the most critical set of requirements identified are the correctness of computed solution and the overall performance. In order to verify those requirements the black-box testing technique was used. The benchmarking results of correctness and performance were based on the requirements specifications. As the black-box technique was used, the final testing for each approach was done once the implementation was finished \citep{blackbox}.

In order to extensively test Alias, benchmark argumentation frameworks have been taken from the published sample argumentation frameworks from ICCMA 2017 \citep{iccmaResults}. The website provides 5 different sets of argumentation frameworks for all tasks involved in the competition and additional set for a Dung's Triathlon. Selected argumentation frameworks have been classified into 

\subsection{Performance}



\subsection{Correctness}