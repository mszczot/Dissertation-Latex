\subsection{Approaches to computing argumentation semantics}



There are many ways of computing abstract argumentation semantics. As shown in section \ref{sec:argumentationSemantics}, semantic definitions can been represented in the form of extensions and labelling. Solvers use different algorithms for computing the extensions. In this section, different approaches for computing abstract argumentation semantics used in different solvers will be reviewed. There are two main factors that will be taken into consideration: speed - how quickly the solver can compute the answer, and correctness - if the results produced are the correct answer.

International Competition on Computational Models of Argumentation is held every 2 years, where different solvers compete on reasoning tasks in abstract argumentation frameworks. The 2017 competition results will be used throughout the project as a point of reference for benchmarking the proposed solution. The competition consist of 7 main tracks, where each track represent each semantics: complete, preferred, stable, semi-stable, stage, grounded and ideal. Furthermore, each track is divided into 4 reasoning problems, with exception for grounded and ideal extensions, where only tasks 1 and 3 are relevant \citep{ICCMA2017}:
\begin{enumerate}
	\item{Given an abstract argumentation framework, determine some extensions}
	\item{Given an abstract argumentation framework, determine all extensions}
	\item{Given an abstract argumentation framework and some argument, decide whether the given argument is credulously inferred}
	\item{Given an abstract argumentation framework and some argument, decide whether the given argument is skeptically inferred}
\end{enumerate}
Each above task consists of 350 benchmark sets divided into 5 categories of hardness from very easy to too hard and each set has a timeout limit of 10 minutes. For each benchmark set the solver can get following scores \citep{results_sildes}:
\begin{itemize}
	\item{1 point, if the output is correct}
	\item{-5 points, if the output is incorrect}
	\item{0 points otherwise, i.e. no result produced within the 10 minutes limit}
\end{itemize}


\subsubsection{Alias}
"ALIAS is a Python library for constructing, manipulating, storing, visualising, and converting argumentation structues" \citep{alias}. It allows to compute the three extensions: complete, preferred and stable, and to build the labellings for complete, grounded, preferred, stable and semi-stable semantics. Since Alias is implemented purely in Python, it can be used as a stand alone tool or a programming library. 

Testing of Alias shown that it is a great tool for computing the abstract argumentation semantics on the smaller argumentation framework. In order to calculate the extensions it generates the power sets of all arguments, and checks whether each individual set is a part of the solution. This approach ensures the correct answer is produced every time, as every possible combination is examined, however, it causes problems with argument frameworks of size larger than twenty arguments. Generating the power set of all arguments and iterating through each possible combination is resource intensive and time consuming. 

% test results of alias

\subsubsection{Pyglaf}

\begin{figure}
	\centering
	\begin{tikzpicture}
	\begin{axis}[
	ybar,
	symbolic x coords={pyglaf, cegartix, argmat-sat, argmat-dvisat, CoQuiAAS, argmat-mpg, goDIAMOND, heureka, conarg, ArgTools, ArgSemSAT, EqArgSolver, argmat-clpb, gg-sts},
	xtick=data,
	x tick label style={rotate=90,anchor=east},
	legend style={at={(0.05,0.1)},anchor=west},
	]
	\addplot table[x=Solver,y=Score]{\completeResults};
	\addplot[draw=red,ultra thick,smooth] table[x=Solver,y=Time]{\completeResults};
	\legend{Score,Time}
	\end{axis}
	\end{tikzpicture}
	\caption{Results of Complete Extension Track}
	\label{fig:coTrack}
\end{figure}

\begin{figure}
	\centering
	\begin{tikzpicture}
	\begin{axis}[
	ybar,
	symbolic x coords={pyglaf,argmat-dvisat,argmat-sat,goDIAMOND,cegartix,ArgTools,argmat-mpg,conarg,CoQuiAAS,gg-sts
	},
	xtick=data,
	x tick label style={rotate=90,anchor=east},
	legend style={at={(0.05,0.1)},anchor=west},
	]
	\addplot table[x=Solver,y=Score]{\idealResults};
	\addplot[draw=red,ultra thick,smooth] table[x=Solver,y=Time]{\idealResults};
	\legend{Score,Time}
	\end{axis}
	\end{tikzpicture}
	
	\caption{Results of Ideal Extension Track}
	\label{fig:idTrack}
	
\end{figure}

\begin{figure}
	\centering
	\begin{tikzpicture}
	\begin{axis}[
	ybar,
	symbolic x coords={pyglaf,goDIAMOND,argmat-sat,cegartix,argmat-mpg,argmat-dvisat,conarg,heureka,ArgSemSAT,ArgTools,EqArgSolver,argmat-clpb,ChimaerArg,CoQuiAAS,gg-sts
	},
	xtick=data,
	x tick label style={rotate=90,anchor=east},
	legend style={at={(0.05,0.1)},anchor=west},
	]
	\addplot table[x=Solver,y=Score]{\stableResults};
	\addplot[draw=red,ultra thick,smooth] table[x=Solver,y=Time]{\stableResults};
	\legend{Score,Time}
	\end{axis}
	\end{tikzpicture}
	
	\caption{Results of StableExtension Track}
	\label{fig:stTrack}
\end{figure}
Pyglaf is one of the solvers submitted to ICCMA 2017 competition. It is one of the best performing solvers in the competition, as it won tracks for Ideal, Complete and Stable extension and had the most winnings of all submitted solvers.

Pyglaf takes advantage of circumscription, a form of non-monotonic reasoning augmenting ordinary first order logic created by \citet{circumpscription}, to formalize the "common sense" assumptions, to solve computational problems of abstract argumentation frameworks.  Circumscriptino, the main solver, is written in C/C++, and is a circumscription solver extending the SAT solver Glucose. However, it uses Python to build the encodings for each semantic and to orchestrate calls to the external solver \citep{pyglaf}. 

As mentioned above, Pyglaf won 3 tracks in ICCMA 2017: complete, stable and ideal extension tracks. As can be seen in figures \ref{fig:coTrack}, \ref{fig:idTrack}, \ref{fig:stTrack}, Pyglaf not only scored the most points for each of them, but also had one of the shortest execution times while delivering correct solutions.

\subsubsection{argmat-sat}
Argmat-sat is another solver from the ICCMA 2017 competition and the winner of Semi-Stable and Stage tracks. It is implemented in C++ and using CryptoMiniSat5 as its SAT engine \citep{argmatSat}.


\subsubsection{Cegartix}
