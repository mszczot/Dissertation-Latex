\subsection{Approaches to computing argumentation semantics} \label{approaches}



There are many ways of computing abstract argumentation semantics. As shown in section \ref{sec:argumentationSemantics}, semantic definitions can been represented in the form of extensions and labelling. Solvers use different algorithms for computing the extensions. In this section, different approaches for computing abstract argumentation semantics used in different solvers will be reviewed. There are two main factors that will be taken into consideration: speed - how quickly the solver can compute the answer, and correctness - if the results produced are the correct answer.

International Competition on Computational Models of Argumentation is held every 2 years, where different solvers compete on reasoning tasks in abstract argumentation frameworks. The 2017 competition results will be used throughout the project as a point of reference for benchmarking the proposed solution. The competition consist of 7 main tracks, where each track represent each semantics: complete, preferred, stable, semi-stable, stage, grounded and ideal. Furthermore, each track is divided into 4 reasoning problems, with exception for grounded and ideal extensions, where only tasks 1 and 3 are relevant \citep{ICCMA2017}:
\begin{enumerate}
	\item{Given an abstract argumentation framework, determine some extensions}
	\item{Given an abstract argumentation framework, determine all extensions}
	\item{Given an abstract argumentation framework and some argument, decide whether the given argument is credulously inferred}
	\item{Given an abstract argumentation framework and some argument, decide whether the given argument is skeptically inferred}
\end{enumerate}
Each above task consists of 350 benchmark sets divided into 5 categories of hardness from very easy to too hard and each set has a timeout limit of 10 minutes. For each benchmark set the solver can get following scores \citep{results_sildes}:
\begin{itemize}
	\item{1 point, if the output is correct}
	\item{-5 points, if the output is incorrect}
	\item{0 points otherwise, i.e. no result produced within the 10 minutes limit}
\end{itemize}


\subsubsection{Alias}
"ALIAS is a Python library for constructing, manipulating, storing, visualising, and converting argumentation structues" \citep{alias}. It allows to compute the three extensions: complete, preferred and stable, and to build the labellings for complete, grounded, preferred, stable and semi-stable semantics. Since Alias is implemented purely in Python, it can be used as a stand alone tool or a programming library. 



Testing of Alias shown that it is a great tool for computing the abstract argumentation semantics on the smaller argumentation framework. In order to calculate the extensions it generates the power sets of all arguments, and checks whether each individual set is a part of the solution. This approach ensures the correct answer is produced every time, as every possible combination is examined, however, it causes problems with argument frameworks of size larger than twenty arguments. Generating the power set of all arguments and iterating through each possible combination is resource intensive and time consuming. 

% test results of alias



\subsubsection{Pyglaf}

\begin{figure}
	\centering
	\begin{tikzpicture}
	\begin{axis}[
	ybar,
	symbolic x coords={pyglaf, cegartix, argmat-sat, argmat-dvisat, CoQuiAAS, argmat-mpg, goDIAMOND, heureka, conarg, ArgTools, ArgSemSAT, EqArgSolver, argmat-clpb, gg-sts},
	xtick=data,
	x tick label style={rotate=90,anchor=east},
	legend style={at={(0.05,0.1)},anchor=west},
	]
	\addplot table[x=Solver,y=Score]{\completeResults};
	\addplot[draw=red,ultra thick,smooth] table[x=Solver,y=Time]{\completeResults};
	\legend{Score,Time}
	\end{axis}
	\end{tikzpicture}
	\caption{Results of Complete Extension Track from ICCMA 2017}
	\label{fig:coTrack}
\end{figure}

\begin{figure}
	\centering
	\begin{tikzpicture}
	\begin{axis}[
	ybar,
	symbolic x coords={pyglaf,argmat-dvisat,argmat-sat,goDIAMOND,cegartix,ArgTools,argmat-mpg,conarg,CoQuiAAS,gg-sts
	},
	xtick=data,
	x tick label style={rotate=90,anchor=east},
	legend style={at={(0.05,0.1)},anchor=west},
	]
	\addplot table[x=Solver,y=Score]{\idealResults};
	\addplot[draw=red,ultra thick,smooth] table[x=Solver,y=Time]{\idealResults};
	\legend{Score,Time}
	\end{axis}
	\end{tikzpicture}
	
	\caption{Results of Ideal Extension Track from ICCMA 2017}
	\label{fig:idTrack}
	
\end{figure}

\begin{figure}
	\centering
	\begin{tikzpicture}
	\begin{axis}[
	ybar,
	symbolic x coords={pyglaf,goDIAMOND,argmat-sat,cegartix,argmat-mpg,argmat-dvisat,conarg,heureka,ArgSemSAT,ArgTools,EqArgSolver,argmat-clpb,ChimaerArg,CoQuiAAS,gg-sts
	},
	xtick=data,
	x tick label style={rotate=90,anchor=east},
	legend style={at={(0.05,0.1)},anchor=west},
	]
	\addplot table[x=Solver,y=Score]{\stableResults};
	\addplot[draw=red,ultra thick,smooth] table[x=Solver,y=Time]{\stableResults};
	\legend{Score,Time}
	\end{axis}
	\end{tikzpicture}
	
	\caption{Results of Stable Extension Track from ICCMA 2017}
	\label{fig:stTrack}
\end{figure}
Pyglaf is one of the solvers submitted to ICCMA 2017 competition. It is one of the best performing solvers in the competition, as it won tracks for Ideal, Complete and Stable extension and had the most winnings of all submitted solvers.

Pyglaf takes advantage of circumscription, a form of non-monotonic reasoning augmenting ordinary first order logic created by \citet{circumpscription}, to formalize the "common sense" assumptions, to solve computational problems of abstract argumentation frameworks.  Circumscriptino, the main solver, is written in C/C++, and is a circumscription solver extending the SAT solver Glucose \citep{glucose}. However, it uses Python to build the encodings for each semantic and to orchestrate calls to the external solver \citep{pyglaf}. 

As mentioned above, Pyglaf won 3 tracks in ICCMA 2017: complete, stable and ideal extension tracks. As can be seen in figures \ref{fig:coTrack}, \ref{fig:idTrack}, \ref{fig:stTrack}, Pyglaf not only scored the most points for each of them, but also had one of the shortest execution times while delivering correct solutions.

\paragraph{Setup}
Pyglaf can be obtained from Mark Alviano's website \citep{Pyglafwebsite}. The direct link provides Pyglaf but also build file to required to build the Circumscriptino. 

Pyglaf provides a standard command-line interface that allows to compute 7 extensions, with option to define whether all or just some of the sets in the extensions should be outputted. It also allows to verify if the provided argument is credulously or skeptically accepted in the given argumentation framework for specific extensions. Finally, it also can compute all grounded, stable and preferred extensions in single call as part of the Dung's Triathlon task. All the functionality has been provided to conform to the ICCMA 2017 competition requirements. 

\subsubsection{argmat-sat}
Argmat-sat developed by \citet{argmatSat} along with argmat-dvisat: division based algorithm framework \citep{argmatDvisat} and argmat-mpg: geocode based solver is another solver submitted to ICCMA 2017 competition. As can be seen in figures \ref{fig:ssTrack} and \ref{fig:stTrack}, Argmat-sat won semi-stable and stage extension tracks significantly outperforming it rivals. It was not only the highest scoring solver for those tracks, but also the fastest. Although argmat-sat won only two tracks, it still was in top 3 solvers for the remaining tracks.

Argmat-sat is implemented purely in C++, using CryptoMiniSAT5 \citep{CryptoMiniSat} as its SAT engine. Similarly to pyglaf \citep{pyglaf}, argmat-sat provides a command line interface that allows to compute all 7 semantics plus Dung Triathlon, to conform to ICCMA 2017 competition requirements. It uses three different CNF encodings: stable extensions, admissible sets and complete extensions, where stable extensions encoding is purely used for computing stable extensions. Admissible sets and complete extensions encodings can be used for searching admissible, complete, preferred, grounded and ideal extensions \citep{argmatSat}. 

In order to compute preferred extension, argmat-sat introduced the 'assumption space' to SAT solvers. Used as a temporary clause base it aids the SAT solver with searching for the maximal extension. Once it is found, the assumption space is cleared to allow for new maximal extension to be searched for. Furthermore, the assumption space is used for computing the semi-stable and stage semantics \citep{argmatSat}.

In order to build and setup argmat-sat additional software is required: cmake, valgrind, libm4ri-dev

\begin{figure}
	\centering
	\begin{tikzpicture}
	\begin{axis}[
	ybar,
	symbolic x coords={argmat-sat,ArgSemSAT,cegartix,pyglaf,goDIAMOND,argmat-mpg,conarg,ArgTools,gg-sts,CoQuiAAS
	},
	xtick=data,
	x tick label style={rotate=90,anchor=east},
	legend style={at={(0.05,0.1)},anchor=west},
	]
	\addplot table[x=Solver,y=Score]{\semiStableResults};
	\addplot[draw=red,ultra thick,smooth] table[x=Solver,y=Time]{\semiStableResults};
	\legend{Score,Time}
	\end{axis}
	\end{tikzpicture}
	
	\caption{Results of Semi Stable Extension Track for ICCMA 2017}
	\label{fig:ssTrack}
\end{figure}


\begin{figure}
	\centering
	\begin{tikzpicture}
	\begin{axis}[
	ybar,
	symbolic x coords={argmat-sat,
		pyglaf,
		cegartix,
		goDIAMOND,
		conarg,
		argmat-mpg,
		ArgTools,
		CoQuiAAS,
		gg-sts,},
	xtick=data,
	x tick label style={rotate=90,anchor=east},
	legend style={at={(0.05,0.1)},anchor=west},
	]
	\addplot table[x=Solver,y=Score]{\stageResults};
	\addplot[draw=red,ultra thick,smooth] table[x=Solver,y=Time]{\stageResults};
	\legend{Score,Time}
	\end{axis}
	\end{tikzpicture}
	
	\caption{Results of Stage Extension Track for ICCMA 2017}
	\label{fig:stTrack}
\end{figure}



\begin{figure}
	\centering
	\begin{tikzpicture}
	\begin{axis}[
	ybar,
	symbolic x coords={ArgSemSAT,
		argmat-sat,
		pyglaf,
		argmat-dvisat,
		cegartix,
		goDIAMOND,
		ArgTools,
		conarg,
		argmat-mpg,
		heureka,
		EqArgSolver,
		ASPrMin,
		ChimaerArg,
		CoQuiAAS,
		gg-sts,},
	xtick=data,
	x tick label style={rotate=90,anchor=east},
	legend style={at={(0.05,0.1)},anchor=west},
	]
	\addplot table[x=Solver,y=Score]{\preferredResults};
	\addplot[draw=red,ultra thick,smooth] table[x=Solver,y=Time]{\preferredResults};
	\legend{Score,Time}
	\end{axis}
	\end{tikzpicture}
	
	\caption{Results of Preferred Extension Track for ICCMA 2017}
	\label{fig:prTrack}
\end{figure}

\subsubsection{Cegartix}

\subsubsection{ArgTools}
Another worth mentioning solver for abstract argumentation semantics is ArgTools. Although it took part in the ICCMA 2017 competition, it did not win any of the tracks included. 

ArgTools is using a different approach to computing argumentation semantics as compared to Pyglaf \citep{pyglaf} or Argmat-Sat \citep{argmatSat}. Instead of encoding the extension with help of the existing SAT solvers, it is using the DPLL (Davis-Putnam-Logemann-Loveland) backtracking algorithm from SAT solving \citep{} to traverse an abstract binary tree created from the definition of the argumentation framework and labels them accordingly. 

ArgTools is implemented purely in C++ and requires at least C++11 in order to be compiled \citep{argtools}. In terms of functionality, the solver adheres to the requirements of the ICCMA competition and can solve all the tasks for complete, stable, preferred, stage, semi-stable, grounded and ideal extensions including computing all or some extensions and also verifying if given argument can be credulously or skeptically accepted for all extensions. Once the ArgTools is build locally, it can be used from the command line, similarly to other solvers, by providing few arguments: location of argumentation framework, format of the file of argumentation framework, task to be performed, and argument to be checked if required. 


