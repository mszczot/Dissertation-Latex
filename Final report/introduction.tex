\section{Aim and Objectives}
The aim of this project is to create high performance and easy to use solver for computing abstract argumentation framework semantics. A number of solvers already exist, however due to their implementation and technologies they use, they can be challenging to setup. Thus, in order to achieve project aims, the following objectives will need to be completed:
\begin{itemize}
	\item Investigation and familiarization with Abstract Argumentation Theory
	\item Investigation and review of existing solvers and solutions
	\item Implementation of different approaches for computing abstract argumentation semantics
	\item Implementation of Web User Interface to aid usability
	\item Benchmark testing of solver
\end{itemize}

\section{Structure of the dissertation}
This report is using following structure:

\begin{itemize}
	\item Section \ref{section:introduction}: Introduction - this section discusses the overall aim and objectives of the project
	\item Section \ref{section:literatureReview}: Literature \& Technical Review - in this section, different abstract argumentation semantics will be reviewed. Furthermore, existing approaches for computing argumentation extensions will be discussed
	\item Section \ref{section:analysis}: Analysis - this section will provide analysis of the existing solutions in preparation for the design stage of the project
	\item Section \ref{section:design}: Design - this section is used to define and describe the project requirements, discuss existing implementation of ALIAS and development methodology used throughout the project
	\item Section \ref{section:implementation}: Implementation - this section will discuss each approach implemented throughout the project
	\item Section \ref{section:testing}: Testing - this section will provide an overview of different testing methods used on the implemented solutions: from unit to benchmark and functionality testing
	\item Section \ref{section:evaluation}: Evaluation - in this section the proposed solution will be evaluated in terms of the identified functional and non-functional requirements. Furthermore, the final solution will be compared and evaluated against the existing solvers	\item Section \ref{section:futureWork}: Future Work - this section will focus on the work that has not been included in the project and any possible future projects that could extend it
	\item Section \ref{section:conclusion}: Conclusion - the final section of the report will outline the achievement of the initially set requirements and the overall project will be looked at in retrospective 
\end{itemize}
