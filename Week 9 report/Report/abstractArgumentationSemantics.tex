-------------------------------------------------------------------------------\\
\textbf{Points to discuss:}
\begin{enumerate}
	\item{\textit{Argumentation Framework}}
	\item{\textit{Abstract argumentation semantics}}
	\item{\textit{Approaches to computing abstract argumentation semantics}}
	\item{\textit{Review of the solvers}}
\end{enumerate}
-------------------------------------------------------------------------------\\


Argumentation framework is central to the theory of abstract argumentation \cite{baroni2011introduction}. The argumentation framework is defined as a pair of a set of arguments, and a binary relation representing the attack relationship between arguments \cite{dung1995}. 

\theoremstyle{definition}
\begin{definition}{Argumentation Framework}
\label{AFdef}\\
An argumentation framework is a pair \textit{AF} = $<$\textit{AR, attacks}$>$ in which \textit{AR} is a set of finite arguments, and \textit{attacks} is a binary relation on \textit{AR}, hence \textit{attacks} $\subseteq$ \textit{AR} $\times$ \textit{AR}, where \textit{AR} $\times$ \textit{AR} = \{(\textit{a}, \textit{b}) $\vert$ \textit{a} $\in$ \textit{AR} and \textit{b} $\in$ \textit{AR}\}
\end{definition}

In definition \ref{AFdef}, \textit{AR} represents a set of arguments and \textit{attacks} represents set of pairs of arguments (\textit{a, b}), where (\textit{a, b}) $\in$ \textit{attacks}. Each pair of arguments in \textit{attacks} represents two arguments being in conflict. Hence, the arguments \textit{a} and \textit{b} are in conflict and the meaning of of \textit{attacks(a, b)} is that \textit{a} represents an attack against \textit{b} \cite{dung1995}. Hence, it can be said that the set of arguments \textit{AR} is conflict free if there are no arguments \textit{a} and \textit{b} in \textit{AR} such that \textit{A} attack \textit{B} \cite{dung1995}.
\newline
 
 
The argumentation framework can be represented as directed graph where the nodes represent abstract arguments and edges the attack relation. This can be seen in figure \ref{fig:argumentationFrameworkFigure}, where argument a attacks argument b, which in turn attacks argument c. C is also attacked by argument d.

\begin{figure}[h]
	\centering
	\includegraphics[width=\textwidth,keepaspectratio=true]{argumentationFramework.png}
	\caption{Representation fo argumentation framework, from \cite{argumentationFrameworkExample}}
	\label{fig:argumentationFrameworkFigure}
\end{figure}
 
Dung in his paper \cite{dung1995} has also defined notions of \textit{acceptable} and \textit{admissible} arguments, which are as follow:

\begin{definition}{Acceptable argument}
\label{AcceptableArgDef}\\
An argument \textit{a} $\in$ \textit{AR} is said to be \textit{acceptable} with respect to set \textit{S} $\subseteq$ \textit{AR} if and only if for each argument \textit{b} $\in$ \textit{AR} such that (\textit{b}, \textit{a}) $\in$ \textit{attacks}, there are some arguments \textit{c} $\in$ \textit{S} such that (\textit{c}, \textit{b}) $\in$ \textit{attacks}
\end{definition}

Hence it can be said that the argument from the argumentation framework is acceptable with respect to the set only if it is either conflict free, or there exists an argument from the same set that defends given argument. Based on the definition \ref{AcceptableArgDef}, definition of \textit{admissible} set can be concluded:

\begin{definition}{Admissible argument}
\label{AdmissibleArgDef}\\
A conflict-free set of arguments \textit{S} is \textit{admissible} if and only if each argument in \textit{S} is acceptable with respect to \textit{S}.
\end{definition}

That means that the set of arguments can be described as admissible only if it is conflict-free. Which in turn means that no argument in S attacks another argument in S. Those definitions have important role in defining the semantics of abstract argumentation.

Furthermore, Dung in his paper \textit{On the acceptability of arguments and its fundamental role in nonmonotonic reasoning, logic programming and n-person games} introduced the concept of semantics of abstract argumentation. The original concept included complete, stable, preferred and grounded semantics \cite{dung1995}, which was extended by ideal \cite{dung2007computing}, and semi-stable \cite{caminada2006semi} semantics. All the sematics are based around the definitions of acceptable and admissible arguments.

\section{Complete Semantics}


\section{Grounded Semantics}

\section{Preferred Semantics} 
As mentioned above the definitions of acceptable and admissible arguments have important role in defeining the extensions. Specifically to preferred semantics, the notion of admissibility is central \cite{caminada2008gentle}. The preferred extension can be defined as a maximal admissible set of argumentation framework. Based on the fact that empty set is always admissible it can be concluded that every argumentation framework has at least one preferred extension \cite{dung1995}.

\section{Stable Semantics} 
Stable semantics although similar, they have a different approach than preferred extension. The stable extension includes the set of arguments that is conflict free and it attacks every argument that is not part of the set. It can be concluded that each stable extension is in fact a preferred extension. However, not every preferred extension is stable extension.



