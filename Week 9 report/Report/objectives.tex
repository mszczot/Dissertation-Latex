The aim of this project is to implement high performance, scalable library for computing abstract argumentation semantics (complete, preffered, stable, grounded, semi-stable and stage semantics) for given argumentation framework. Furthermore, it should have the ability to compute if the given argument from the given argumentation framework is credulously or skeptically inferred for given semantics.\\
This project can be divided into four separate stages: research, design, development and evaluation. Although the stages proposed can be performed in the sequence, work on this project will be performed in the iterative way. In the work plan for the project, three iterations of desing, development and testing have been identified following a period of research, background reading and writing literature review. However, further research might be required in the later stages.

Each stage of the project have overall goals that will neeed to be achieved. Research and review stage is the vital part of the project. This will allow to build a better understanding of the problem domain and investigate existing solutions. Following goals can be identified for this stage:
\begin{enumerate}
	\item{Research concept of argumentation framework and abstract argumentation semantics}
	\item{Research and review existing algorithms for solving the problem}
	\item{Review solvers submitted to ICCMA 2017}
	\item{Perform technology review}
\end{enumerate}

Once enough research will be performed, the project can proceed to next stages. The knowledge gathered will be used to define the requirements and propose appropriate solution. The design, development and testing stages will be performed in an iterative way. Goals for those stages include:
\begin{enumerate}
	\item{Perform Requirements Analysis}
	\item{Define Test Plan}
	\item{Based on the requirements analysis design appropriate solution} 
	\item{Implement the proposed solution}
	\item{Test the solution to verify implemented requirements}
	\item{Perform benchmark testing to evaluate the performance and correctness of the solution}
\end{enumerate}
As mentioned above, 3 iterations of those stages have been initially scheduled. Additionaly, during any of those stages further research might be required, hence it might be needed to go back to research stage.
\newline
The outcome of the project will be the developed library for computing abstract argumentation framework. Furthermore, depending on the evaluation process, the library will be used to develop a solver for abstract argumentation framework to be submitted to International Competition on Computational Models of Argumentation \citep{ICCMA}. Hence, there are number of requirements that will have to be included in the implementation. Below is the list of the main requirements, but this list is not extensive and will be expanded during the course of a project:
\begin{enumerate}
	\item{Should be able to read argumentation from provided file in format apx or tgf}
	\item{Should be able to compute the following semantics from the given argumentation framework and return all or some of the extensions computed:}
		\begin{enumerate}
			\item{Complete Semantics}
			\item{Preferred Semantics}
			\item{Stable Semantics}
			\item{Semi-Stable Semantics}
			\item{Stage Semantics}
			\item{Grounded Semantics}
			\item{Ideal Semantics}
		\end{enumerate}
	\item{Should take the type of task to be computed as an argument}
\end{enumerate}
The above list shows only the basic requirements for the software. The extended list can be found in Appendix \ref{appendix:requirementAnalysis}. Although this list is more comprehensive, during the course of the project, new requirements will be identified. Hence, this list will be a subject to regular updates, based on the progress of the project. Before each development stage, a set of requirements will be selected from the list. \\
The above requirements are functional requirements of the software. Additionaly, there are a number of non-functional requirements that will have to be included in the design and implementation software. Those requirements are needed if the software will be submitted to the ICCMA competition. 