\section{Argumentation Semantics}
\label{sec:argumentationSemantics}


The original concept of abstract argumentation semantics included complete, stable, preferred and grounded extensions \citep{dung1995}, which was extended by stage \citep{verheij1996two}, ideal \citep{dung2007computing}, and semi-stable \citep{caminada2006semi} semantics. All the sematics are based around the definitions of acceptable and admissible arguments. Definitions of the original semantics as described by \citet{dung1995} are as follow: 

\begin{definition}{Preferred Extension}
\label{def:preferredExtension}\\
A \textit{preferred extension} of an argumentation framework \textit{AF} is a maximal (with respect to set inclusion) admissible set of \textit{AF}.
\end{definition}


From the definition \ref{def:preferredExtension} it can be seen that the preferred extension is the largest set that is able to defend itself from attacks. If we consider example \ref{fig:af1} we can see that the extension \textit{$E_{PR}$} = \{\{a,c\}, \{b\}\}. Argument \textit{c} is attacked by argument \textit{b}, which in turn is defeated by \textit{a}, hence \textit{a} and \textit{c} are preferred extension. Furthermore, \textit{b} is attacked by \textit{a}, but it defends itself. Hence argument \textit{b} is also preferred extension.

\begin{figure}[h]
\tikzset{
    main/.style={draw, rectangle, rounded corners, minimum height=1cm, minimum width=4.5cm},
    arrow/.style={thick,<-,>=stealth}
}
\centering
\begin{tikzpicture}[auto,node distance=1.5cm]
	\node[draw=none,fill=none](c){c};
	\node[draw=none,fill=none][right=of c](b){b};
	\node[draw=none,fill=none][right=of b](a){a};	
  	%%% ARROWS %%%
  	\draw[thick,<-,>=stealth,transform canvas={yshift=0.5em}](b) -- (a);
  	\draw[arrow](c) -- (b);
  	\draw[thick,<-,>=stealth,transform canvas={yshift=-0.2em}](a) -- (b);
\end{tikzpicture}
\caption{Argumentation Framework \ref{fig:af1}}
\label{fig:af1}
\end{figure}

\begin{definition}{Stable Extension}
\label{def:stableExtension}\\
A \textit{conflict-free} set of arguments \textit{S} is called a \textit{stable extension} if and only if \textit{S} attacks each argument which does not belong to \textit{S}.
\end{definition}

Stable extension has more aggressive approach than preferred extension. As in definition \ref{def:stableExtension}, the extension needs to attack all arguments that are not included in it. Based on argumentation framework from example \ref{fig:af1}, the stable extension \textit{$E_{ST}(AF_{\ref{fig:af1}})$} = \{\{a,c\}, \{b\}\}, which is the same as for preferred extension. This is because argument \textit{a} attacks \textit{b}, hence \textit{a} and \textit{c} are included, and argument \textit{b} attack both arguments \textit{a} and \textit{c}. Looking at more complex example in figure \ref{fig:af2}, three stable extensions can be identified as follow \textit{$E_{ST}(AF_{\ref{fig:af2}})$} = \{\{a,c\}, \{a,d\}, \{b,d\}\}.

\begin{figure}[h]
\tikzset{
    main/.style={draw, rectangle, rounded corners, minimum height=1cm, minimum width=4.5cm},
    arrow/.style={thick,<-,>=stealth}
}
\centering
\begin{tikzpicture}[auto,node distance=1.5cm]
	\node[draw=none,fill=none](c){c};
	\node[draw=none,fill=none][left=of c](d){d};
	\node[draw=none,fill=none][right=of c](b){b};
	\node[draw=none,fill=none][right=of b](a){a};	
  	%%% ARROWS %%%
  	\draw[thick,<-,>=stealth,transform canvas={yshift=0.5em}](b) -- (a);
  	\draw[arrow](c) -- (b);
  	\draw[thick,<-,>=stealth,transform canvas={yshift=-0.2em}](a) -- (b);
  	\draw[thick,<-,>=stealth,transform canvas={yshift=-0.2em}](c) -- (d);
  	\draw[thick,<-,>=stealth,transform canvas={yshift=0.5em}](d) -- (c);
\end{tikzpicture}
\caption{Argumentation Framework \ref{fig:af2}}
\label{fig:af2}
\end{figure}

\citet{dung1995} in his paper also defined a skeptical semantics known as grounded extension. This extension is defined in terms of \textit{characteristic function}, which in turn is defined as:

\begin{definition}{Characteristic Function}
\label{def:characteristicFunction}\\
The \textit{characteristic function}, denoted by \textit{F\textsubscript{AF}}, of an argumentation framework \textit{AF} = $<$\textit{AR, attacks}$>$ is defined as follows:
\[F_{AF}:2^{AR} \rightarrow 2^{AR}\]
\[F_{AF}(S)=\{A| \text{A is acceptable with respect to S} \}\]
\end{definition}

\begin{definition}{Grounded Extension}
\label{def:groundedExtension}\\
The \textit{grounded extension} of an argumentation framework \textit{AF}, denoted by \textit{GE\textsubscript{AF}}, is the least fixed point of \textit{F\textsubscript{AF}}.
\end{definition}

Based on the definition of characteristic function and grounded extension it can be noted that the grounded extension has very restrictive approach for its computation, as it only includes the arguments whose defence is "rooted" in initial unattacked argument \citep{baroni2009semantics}. Hence, the grounded extension from figure \ref{fig:af3} is \textit{$E_{GR}(AF_{\ref{fig:af3}})$} = \{\{a,c\}\}. Starting from argument \textit{a}, as it is the only unattacked argument, we reject argument \textit{b} and include \textit{c}. Furthermore, argument \textit{d} is being rejected, as it is attacked by \textit{c}. Although, argument \textit{e} attacks argument \textit{f}, it is attacked by \textit{f} as well. Hence, it cannot be part of grounded extension.

\begin{figure}[h]
\tikzset{
    main/.style={draw, rectangle, rounded corners, minimum height=1cm, minimum width=4.5cm},
    arrow/.style={thick,<-,>=stealth}
}
\centering
\begin{tikzpicture}[auto,node distance=1.5cm]
	\node[draw=none,fill=none](a){a};
	\node[draw=none,fill=none][right=of a](b){b};
	\node[draw=none,fill=none][right=of b](c){c};	
	\node[draw=none,fill=none][right=of c](d){d};
	\node[draw=none,fill=none][right=of d](e){e};
	\node[draw=none,fill=none][right=of e](f){f};			
  	%%% ARROWS %%%
  	\draw[arrow](b) -- (a);
  	\draw[arrow](c) -- (b);
  	\draw[arrow](d) -- (c);
  	\draw[arrow](e) -- (d);
  	\draw[thick,<-,>=stealth,transform canvas={yshift=-0.2em}](e) -- (f);
  	\draw[thick,<-,>=stealth,transform canvas={yshift=0.5em}](f) -- (e);
\end{tikzpicture}
\caption{Argumentation Framework \ref{fig:af3}}
\label{fig:af3}
\end{figure}

\begin{definition}{Complete Extension}
\label{def:completeExtension}\\
An admissible set \textit{S} of arguments is called \textit{complete exension} if and only if each argument which is acceptable with respect to \textit{S}, belongs to \textit{S}.
\end{definition}

Complete extension is defined as a set which is able to defend itself and includes all arguments it defends \citep{baroni2009semantics}. Based on this, we can see that there are three complete extensions in figure \ref{fig:af1}, and those are \textit{$E_{CO}(AF_{\ref{fig:af1}})$} = \{$\emptyset$, \{a,c\}, \{b\}\}. With a more complex argumentation framework in figure \ref{fig:af2}, it can be seen that the complete extension will definitely include empty set, as there are no initial arguments, argument \textit{a}, as it does not defend argument \textit{c} from \textit{d}, and argument \textit{d}, as it only defends itself. Furthermore, \textit{b} attacks \textit{c}, the only attacked of argument \textit{d}, hence set \{b,d\} is a complete extension as well. Additionally, sets \{a,c\} and \{a,d\} are also compltete extensions as both of them are admissible. Hence, the complete extension is \textit{$E_{CO}(AF_{\ref{fig:af2}})$} = \{$\emptyset$, \{a\}, \{d\}, \{a,c\}\, \{a,d\}, \{b,d\}\}.

As mentioned earlier, \citet{dung1995} semantics were further extended with additional semantics. Their definitions are as follow: 

\begin{definition}{Semi-stable Extension}
\label{def:semiStableExtension}\\
Let (\textit{AR}, \textit{attacks}) be an argumentation framework and \textit{Args} $\subseteq$ \textit{AR}. \textit{Args} is called \textit{semi-stable extension} if and only if \textit{Args} is a complete extension where \textit{Args} $\cup$ \textit{$Args^+$} is maximal.
\end{definition}

The idea for semi-stable extension consist in expressing a definite opinion on the largest possible set of arguments, while restricting as much as possible those which are left undecided \citep{baroni2011introduction}, where labelling has been used. Hence, the stable-extension in argumentation framework in figure \ref{fig:af4} is \textit{$E_{SST}(AF_{\ref{fig:af4}})$} = \{\{b,d\}\}, which is also a preferred extension.

\begin{figure}[h]
\tikzset{
    main/.style={draw, rectangle, rounded corners, minimum height=1cm, minimum width=4.5cm},
    arrow/.style={thick,<-,>=stealth}
}
\centering
\begin{tikzpicture}[auto,node distance=1.5cm]
	\node[draw=none,fill=none](a){a};
	\node[draw=none,fill=none][right=of a](b){b};
	\node[draw=none,fill=none][right=of b](c){c};	
	\coordinate[right=0.75cm of c](coordinate);
	\node[draw=none,fill=none][above =0.5cm of coordinate](d){d};
	\node[draw=none,fill=none][below =0.5cm of coordinate](e){e};		
  	%%% ARROWS %%%
  	\draw[thick,<-,>=stealth,transform canvas={yshift=-0.2em}](a) -- (b);
  	\draw[thick,<-,>=stealth,transform canvas={yshift=0.5em}](b) -- (a);
  	\draw[arrow](c) -- (b);
	\draw[thick, -latex] (c.north) to [bend left=45](d.west);
	\draw[thick, -latex] (d.east) to [bend left=90](e.east);
	\draw[thick, -latex] (e.west) to [bend left=45](c.south);
\end{tikzpicture}
\caption{Argumentation Framework \ref{fig:af4}}
\label{fig:af4}
\end{figure}

\begin{definition}{Ideal Extension}
\label{def:idealExtension}\\
The \textit{ideal extension} is the greatest with respect to set inclusion admissible set that is a subset of each preferred extension.
\end{definition}

By its definition the ideal extension satisfies I-maximality and admissibility principles \citep{baroni2009semantics}. The I-maximiality is a property of the set of extensions, without reference to any generic criterion \citep{dunne2006computational}. It is defined as:

\begin{definition}{I-maximality}
\label{def:IMaximality}\\
A set of extensions \textit{E} is I-maximal if and only if $\forall$\textit{$E_1$}, \textit{$E_2$} $\in$ \textit{E}, if \textit{$E_1$} $\subseteq$ \textit{$E_2$} then \textit{$E_1$} = \textit{$E_2$}.
\end{definition}

\begin{definition}{Stage Extension}
\label{def:stageExtension}\\
Let (\textit{AR}, \textit{attacks}) be an argumentation framework. A \textit{stage extension} conflict-free set \textit{Args} $\subseteq$ \textit{AR}, where \textit{Args} $\cup$ \textit{$Args^+$} is maximal with respect to set inclusion among all conflict-free sets.
\end{definition}


\begin{figure}[h]
\tikzset{
    main/.style={draw, rectangle, rounded corners, minimum height=1cm, minimum width=4.5cm},
    arrow/.style={thick,<-,>=stealth}
}
\centering
\begin{tikzpicture}[auto,node distance=1.5cm]
	\node[main](1){Complete Semantics};
	\coordinate[below=of 1](c);
  	\node[main](2)[left=of c]{Grounded Semantics};
  	\node[main](3)[right=of c]{Ideal Semantics};
  	\node[main](4)[below=of c]{Preferred Semantics};
  	\node[main](5)[below=of 4]{Semi-Stable Semantics};
  	\node[main](7)[below=of 5]{Stable Semantics};  	
  	\node[main](6)[right=of 7]{Stage Semantics};
  	%%% ARROWS %%%
  	\draw[arrow](1) -- (2) node[midway] {is a};
  	\draw[arrow](1) -- (3) node[midway] {is a};
  	\draw[arrow](1) -- (4) node[midway] {is a};
  	\draw[arrow](4) -- (5) node[midway] {is a};
  	\draw[arrow](5) -- (7) node[midway] {is a};  	
  	\draw[arrow](6) -- (7) node[midway] {is a};
\end{tikzpicture}
\caption{Relations of semantics}
\label{fig:semanticsRelations}
\end{figure}

Above semantics can be used to decide whether a given argument, or set of arguments can be accepted in terms of set inclusion.  Based on the definitions a number of inclusions between the sets can be conluded. Those are:
\begin{itemize}
	\item{Grounded extension is the smallest complete extension}
	\item{Every preferred extesion is also complete}
	\item{Every stable extension is also semi-stable and stage extension}
\end{itemize}
Furthermore, it can be seen that the complete extension provides the link between preferred extension (credulous semantic), and grounded extension (skeptical smenatics) \citep{dung1995}. It can also be concluded seen that the maximal complete extension is also maximal admissible set. On the other hand the grounded extension is determined by the minimal complete extension \citep{gaggl2009solving}. The relations between the semantics can be seen in figure \ref{fig:semanticsRelations}.


%sceptical and credulous semantics
%sceptical and credulous acceptance of argument
Additional notion of credulous and sceptical semantics has been introduced. Usually, a credulous semantics is intended to capture as much information as possible \citep{dix1995non}, while skeptical makes less committed choices about the justification of the arguments \citep{baroni2007comparing}. As mentioned previosly grounded semantic is a good example of skeptical extension as it always yields exactly one extension, and this extension is admissible \citep{caminada2007comparing}.

Furthermore, argument can be defined as credulous or skeptical in terms of its acceptability. As describe by \citet{arieli2015conflict} an argument \textit{a} $\in$ \textit{Args} is credulously accepted by one type of semantics only if it belongs to some of the semantic extensions of a given argumentation framework. On the other hand, the argument \textit{b} is skeptically accepted by the semantic if it belongs to all of its extensions of a given argumentation framework. This can be represented using argumentation framework \ref{fig:af5}. It has 5 complete extensions: $\emptyset$, {b}, {e}, {b, e}, {a, c, e}, however none of the arguments is skeptically accepted, while \textit{a}, \textit{b}, \textit{c} and \textit{e} are credulously accepted. On the other hand, \{\textit{b, e}\} and \{\textit{a, c, e}\} are preferred, stable, semi-stable and stage extensions of argumentation framework \ref{fig:af5}. It can be seen that only argument \textit{e} is skeptically accepted for all those semantics, while \textit{a}, \textit{b} and \textit{c} are credulously accepted.

\begin{figure}[h]
\tikzset{
    main/.style={draw, rectangle, rounded corners, minimum height=1cm, minimum width=4.5cm},
    arrow/.style={thick,<-,>=stealth}
}
\centering
\begin{tikzpicture}[auto,node distance=1.5cm]
	\node[draw=none,fill=none](a){a};
	\node[draw=none,fill=none][right=of a](b){b};
	\node[draw=none,fill=none][right=of b](c){c};	
	\node[draw=none,fill=none][right=of c](d){d};	
	\coordinate[right=0.75cm of d](coordinate);
	\node[draw=none,fill=none][above =0.5cm of coordinate](e){e};
	\node[draw=none,fill=none][below =0.5cm of coordinate](f){f};		
  	%%% ARROWS %%%
  	\draw[thick,<-,>=stealth,transform canvas={yshift=-0.2em}](a) -- (b);
  	\draw[thick,<-,>=stealth,transform canvas={yshift=0.5em}](b) -- (a);
  	\draw[arrow](c) -- (b);
  	\draw[arrow](d) -- (c);
	\draw[thick, -latex] (d.north) to [bend left=45](e.west);
	\draw[thick, -latex] (e.east) to [bend left=90](f.east);
	\draw[thick, -latex] (f.west) to [bend left=45](d.south);
\end{tikzpicture}
\caption{Argumentation Framework \ref{fig:af5}}
\label{fig:af5}
\end{figure}